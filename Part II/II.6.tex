\documentclass[main.tex]{subfiles}
% 基变换与坐标变换
\begin{document}
本节讨论同一个向量或线性变换在不同基下的矩阵之间的关系。

考虑数域$\mathbb{F}$上的$N$维向量空间$\mathcal{V}_N$中的两组基$\left\{\mathbf{e}_i\right\},\left\{\mathbf{e}^\prime_j\right\}$。用第一组基表示第二组基的每个基向量,可列出如下的$N$个等式:
\[\mathbf{e}^\prime_j=\sum_{i=1}^NS_{ij}\mathbf{e}_i,j=1,\cdots,N\]
称为从基$\left\{\mathbf{e}_i\right\}$到基$\left\{\mathbf{e}^\prime_j\right\}$的基变换公式。矩阵$\left(S_{ij}\right)$称为从基$\left\{\mathbf{e}_i\right\}$到基$\left\{\mathbf{e}^\prime_j\right\}$的过渡矩阵。

\begin{theorem}
设$\mathcal{V}_N$是数域$\mathbb{F}$上的$N$维向量空间,$\left\{\mathbf{e}_i\right\},\left\{\mathbf{e}^\prime_j\right\}$是$\mathcal{V}_N$的两组基,则从基$\left\{\mathbf{e}_i\right\}$到基$\left\{\mathbf{e}^\prime_j\right\}$的过渡矩阵是从基$\left\{\mathbf{e}^\prime_j\right\}$到基$\left\{\mathbf{e}_i\right\}$的过渡矩阵的逆矩阵。若$\mathbf{e}^\prime_j=\sum_{i=1}^NS_{ij}\mathbf{e}_i,\mathbf{e}_i=\sum_{i=1}^NT_{ji}\mathbf{e}^\prime_j$,则$S=T^{-1}$。
\end{theorem}
\begin{proof}
如果我们把基向量$\left\{\mathbf{e}_i\right\}$排成一个列向量,则基变换关系可表示为\footnote{虽然这严格来说不是一个矩阵,但矩阵乘以向量的式子本质上是线性方程组的一种表示,故此入可视为一个线性方程组的简化表示。矩阵的互逆,本质上也是线性方程组的求解,故本命题也可通过矩阵运算法则得到的互逆性得以证明。}:
\[
\left(\begin{array}{c}
\mathbf{e}^\prime_1\\
\vdots\\
\mathbf{e}^\prime_N
\end{array}\right)=\left(\begin{array}{ccc}
S_{11}&\cdots&S_{1N}\\
\vdots&\ddots&\vdots\\
S_{N1}&\cdots&S_{NN}
\end{array}\right)\left(\begin{array}{c}
\mathbf{e}_1\\
\vdots\\
\mathbf{e}_N
\end{array}\right)
\]
由此易见$S=T^{-1}$。
\end{proof}

特别地,由$n$维向量空间的一组基到它自身的过渡矩阵是单位矩阵$I_n$。

我们通过基的过渡矩阵,可以写出一个向量$\mathbf{v}\in\mathcal{V}_N$在两组基$\left\{\mathbf{e}_i\right\},\left\{\mathbf{e}^\prime_j\right\}$下的坐标之间的关系:

\begin{align*}
\mathbf{v}&=\sum_{j=1}^Nv^\prime_j\mathbf{e}^\prime_j\\
&=\sum_{j=1}^Nv^\prime_j\left(\sum_{i=1}^NS_{ij}\mathbf{e}_i\right)\\
&=\sum_{i=1}^N\left(\sum_{j=1}^N S_{ij}v^\prime_j\right)\mathbf{e}_i\\
&=\sum_{i=1}^Nv_i\mathbf{e}_i\\
\Leftrightarrow v_i&=\sum_{j=1}^NS_{ij}v^\prime_j,i=1,\cdots,N
\end{align*}
这$N$个式子称为向量$\mathbf{v}$从基$\left\{\mathbf{e}^\prime_j\right\}$到$\left\{\mathbf{e}_i\right\}$的坐标变换公式,也可以写成矩阵乘:
\[\left(\begin{array}{c}v_1\\\vdots\\v_N\end{array}\right)=\left(\begin{array}{ccc}S_{11}&\cdots&S_{1N}\\\vdots&&\vdots\\S_{N1}&\cdots&S_{NN}\end{array}\right)\left(\begin{array}{c}v^\prime_1\\\vdots\\v^\prime_N\end{array}\right)\]

注意到,对于同一个矩阵$S_{ij}$,它是从$\left\{\mathbf{e}^\prime_j\right\}$到$\left\{\mathbf{e}_i\right\}$的过渡矩阵,但却用于向量$\mathbf{v}$从$\left\{\mathbf{e}_i\right\}$下的坐标到$\left\{\mathbf{e}^\prime_j\right\}$下的坐标的变换公式中。按照相同的推算方法还可以得到,向量$\mathbf{v}$从$\left\{\mathbf{e}^\prime_j\right\}$到$\left\{\mathbf{e}_i\right\}$的坐标变换公式是$v^\prime_j=\sum_{i=1}^NT_{ji}v_i,j=1,\cdots,N$,其中$T_{ji}=S_{ij}^{-1}$是从$\left\{\mathbf{e}^\prime_j\right\}$到$\left\{\mathbf{e}_i\right\}$的过渡矩阵。

接下来,我们看线性变换的矩阵在不同基下的变换公式。

\begin{theorem}
设$\mathcal{V}_N$、$\mathcal{W}_M$分别是数域$\mathbb{F}$上的$N$、$M$维向量空间,$\left\{\mathbf{a}_i\right\},\left\{\mathbf{a}^\prime_i\right\}\in\mathcal{V}_N$是$\mathcal{V}_N$的两组基,基变换公式为$\mathbf{a}^\prime_j=\sum_{i=1}^NS_{ij}\mathbf{a}_i$;$\left\{\mathbf{b}_j\right\},\left\{\mathbf{b}^\prime_j\right\}$是$\mathcal{W}_M$的两组基,基变换公式为$\mathbf{b}^\prime_j=\sum_{i=1}^MT_{ij}\mathbf{b}_i$。线性变换$\mathbf{A}:\mathcal{V}_N\rightarrow\mathcal{W}_M$在$\left\{\mathbf{a}_i\right\},\left\{\mathbf{b}_i\right\}$下的矩阵表示为$\left(\mathbf{A}\right)$,在$\left\{\mathbf{a}^\prime_i\right\},\left\{\mathbf{b}^\prime_i\right\}$下的矩阵表示为$\left(\mathbf{A}\right)^\prime$。则有
\begin{align*}
    \left(\mathbf{A}\right)&=T\left(\mathbf{A}\right)^\prime S^{-1}\\
    \left(\mathbf{A}\right)^\prime&=T^{-1}\left(\mathbf{A}\right)S
\end{align*}
\end{theorem}
\begin{proof}
对于任一向量$\mathbf{v}\in\mathcal{V}_N$,$\mathbf{w}=\mathbf{Av}\in\mathcal{W}_M$。我们从向量$\mathbf{w}$的坐标变换出发:
\begin{align*}
    w_i&=\sum_{j=1}^MT_{ij}w^\prime_j\\
    &=\sum_{j=1}^MT_{ij}\left(\sum_{k=1}^NA^\prime_{jk}v^\prime_k\right)\\
    &=\sum_{j=1}^M\sum_{k=1}^NT_{ij}A^\prime_{jk}\left(\sum_{l=1}^NS^{-1}_{kl}v_l\right)\\
    &=\sum_{j=1}^M\sum_{k=1}^N\sum_{l=1}^NT_{ij}A^\prime_{jk}S^{-1}_{kl}v_l\\
    &=\sum_{j=1}^N\sum_{l=1}^M\sum_{k=1}^NT_{il}A^\prime_{lk}S^{-1}_{kl}v_j,\quad i=1,\cdots,M
\end{align*}
其中$A_{ij},A^\prime_{ij}$分别是$\left(\mathbf{A}\right),\left(\mathbf{A}\right)^\prime$的坐标。

另一方面,$w_i=\sum_{j=1}^NA_{ij}v_j,i=1,\cdots,M$,与上面的结果比较可得:
\[
A_{ij}=\sum_{l=1}^M\sum_{k=1}^NT_{il}A^\prime_{lk}S^{-1}_{kj}
\Leftrightarrow \left(\mathbf{A}\right)=T\left(\mathbf{A}\right)^\prime S^{-1}
\]
由$T^{-1}\left(\mathbf{A}\right)S=T^{-1}T\left(\mathbf{A}\right)^\prime S^{-1}S=\left(\mathbf{A}\right)^\prime$,可得$\left(\mathbf{A}\right)^\prime=T^{-1}\left(\mathbf{A}\right)S$。
\end{proof}
\end{document}