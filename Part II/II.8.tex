\documentclass[main.tex]{subfiles}
% 张量与张量积
\begin{document}
张量是比向量更一般的数学概念。张量满足向量的代数定义,故它也是一个抽象向量,类似向量作为向量空间的元素是一个不依赖基的选择的量,张量也是一个不依赖基的选择的量。$N$维向量在任一基下的矩阵是$N\times 1$矩阵,但$n$阶张量在选定任一基下的矩阵是$N^n$矩阵。我们之前介绍的线性算符,就是2阶张量。在连续介质力学中,我们一般只涉及到实数域上的线性算符,同时很少涉及到2阶以上的张量。之前所说的向量都是指1阶张量。我们在本课中仍然用“向量”表达1阶张量的意思,而把2阶张量简称为“张量”。

\begin{definition}[两个向量的张量积]
两个向量$\mathbf{v},\mathbf{w}\in\mathcal{V}$的张量积$\mathbf{v}\otimes\mathbf{w}\in\mathcal{L}\left(\mathcal{V}\right)$是一个二阶张量满足:
\[\left(\mathbf{v}\otimes\mathbf{w}\right)\mathbf{u}=\mathbf{v}\left(\mathbf{w}\cdot\mathbf{u}\right),\forall\mathbf{u}\in\mathcal{V}\]
\end{definition}

易验证张量积满足线性变换的代数规则。

并非每个张量都能表示成两个向量的向量积。我们把$\mathcal{V}$的两个向量的向量积的集合记为$\mathcal{V}\otimes\mathcal{V}$,由张量积的定义易知,$\mathcal{V}\otimes\mathcal{V}$是线性算符的集合。因此$\mathcal{V}\otimes\mathcal{V}$是$\mathcal{L}\left(\mathcal{V}\right)$的子集。

下面的例子给出了两个向量的张量积的坐标计算方法。

\begin{example}
设向量$\mathbf{v},\mathbf{w}\in\mathcal{V}_N$且$\left\{\mathbf{e}_i\right\}$是$\mathcal{V}_N$的一组基,则有:
\[\mathbf{v}=\sum_{i=1}^Nv_i\mathbf{e}_i,\mathbf{w}=\sum_{i=1}^Nw_i\mathbf{e}_i\]
\begin{align*}
\mathbf{v}\otimes\mathbf{w}&=\left(\sum_{i=1}^Nv_i\mathbf{e}_i\right)\otimes\left(\sum_{j=1}^Nw_j\mathbf{e}_j\right)\\&=\sum_{i=1}^N\sum_{j=1}^Nv_iw_j\mathbf{e}_i\otimes\mathbf{e}_j
\end{align*}
\end{example}

由上例,可见两向量的张量积可由基的张量积线性表出。且$\left\{\mathbf{\hat{e}}_i\otimes\mathbf{\hat{e}}_j\right\}$有$N\times N$个。如果它们是线性无关的,那它就是$\mathcal{L}\left(\mathcal{V}\right)$的一组基了。但很可惜,一般地,它们并不一定是线性无关的。回顾定理\ref{thm:II.4.3}证明和上面的张量积定义,就可以直接证明如下定理。

\begin{theorem}
设$\mathcal{V}_N$、$\mathcal{W}_M$分别是数域$\mathbb{F}$上的$N$、$M$维向量空间,$\left\{\mathbf{\hat{e}}_i\right\}$是$\mathcal{V}_N$的一组基,$\left\{\mathbf{\hat{f}}_j\right\}$是$\mathcal{W}_N$的一组基。当且仅当$\left\{\mathbf{\hat{e}}_i\right\}$是规范正交基时,$\left\{\mathbf{\hat{f}}_j\otimes\mathbf{\hat{e}}_i\right\}$是$\mathcal{L}\left(\mathcal{V}_N,\mathcal{W}_M\right)$的基。
\end{theorem}

因此,对于$\mathcal{L}\left(\mathcal{V}\right)$的情况,就是简单地需要$\left\{\mathbf{\hat{e}}_i\right\}$是$\mathcal{V}$的一组规范正交基,$\left\{\mathbf{\hat{e}}_i\otimes\mathbf{\hat{e}}_j\right\}$才是$\mathcal{L}\left(\mathcal{V}\right)$的一组基。

\begin{definition}
两个张量的复合积就是它们作为两个线性变换的复合变换。
\end{definition}

下面介绍的是有限维内积空间$\mathcal{V}$上的张量才有的概念。

\begin{definition}[张量的转置]\footnote{这里定义的“转置”对应的是复数域上线性算符的伴随(adjoint)的概念,后者也仅是复数域上线性算符的转置(transpose)的一个特例。这里使用“转置”一词只是遵循大部分流变学文本的惯例。}
设$\mathcal{V}$是实数域$\mathbb{R}$上的有限维内积空间,张量$\mathbf{A},\mathbf{B}\in\mathcal{L}\left(\mathcal{V}\right)$。若对于$\mathbf{u},\mathbf{v}\in\mathcal{V}$有$\left(\mathbf{Au}|\mathbf{v}\right)=\left(\mathbf{u}|\mathbf{Bv}\right)$,则称$\mathbf{B}$是$\mathbf{A}$的转置,记为$\mathbf{B}\equiv\mathbf{A}^\intercal$。
\end{definition}

\begin{theorem}
任一张量总存在唯一一个转置。
\end{theorem}

\begin{definition}[对称张量与斜称张量]
如果张量$\mathbf{A}=\mathbf{A}^\intercal$,则称$\mathbf{A}$是对称张量。如果$\mathbf{A}^\intercal=-\mathbf{A}$,则称$\mathbf{A}$是斜称张量。
\end{definition}

\begin{theorem}
任何一个张量$\mathbf{A}$均可以写成一个对称张量$\mathbf{A}^\mathrm{s}$与一个斜称张量$\mathbf{A}^\mathrm{ss}$之和:
\[\mathbf{A}=\mathbf{A}^\mathrm{s}+\mathbf{A}^\mathrm{ss}\]
其中
\[\begin{split}\mathbf{A}^\mathrm{s}&=\frac{1}{2}\left(\mathbf{A}+\mathbf{A}^\intercal\right)\\\mathbf{A}^\mathrm{ss}&=\frac{1}{2}\left(\mathbf{A}-\mathbf{A}^\intercal\right)\end{split}\]
\end{theorem}

\begin{definition}[正交张量]\footnote{这里的正交张量对应复数域上的幺正算符(unitary operator)的概念。}
若张量$\mathbf{A}$满足$\mathbf{AA}^\intercal=\mathbf{I}$其中$\mathbf{I}$是单位张量(即恒等算符),则称$\mathbf{A}$是正交张量。
\end{definition}

\begin{definition}[张量的迹]
张量的迹是一个实函数$\mathrm{tr}:\mathcal{L}\left(\mathcal{V}\right)\rightarrow\mathbb{R}$满足:
\begin{itemize}
    \item 线性:$\mathrm{tr}\left(\alpha\mathbf{S}+\beta\mathbf{T}\right)=\alpha\mathrm{tr}\mathbf{S}+\beta\mathrm{tr}\mathbf{T},\forall\alpha,\beta\in\mathbb{R},\mathbf{S},\mathbf{T}\in\mathcal{L}\left(\mathcal{V}\right)$
    \item $\mathrm{tr}\left(\mathbf{u}\otimes \mathbf{v}\right)=\mathbf{u}\cdot\mathbf{v}\forall\mathbf{u},\mathbf{v}\in\mathcal{V}$
\end{itemize}
\end{definition}

可以证明,满足上述条件的迹运算只有一种\cite[\S Appendix]{Coleman1966},且张量的迹不依赖基的选择,特以定理列出。

\begin{theorem}
张量的迹不依赖基的选择。
\end{theorem}
\begin{proof}
把一个张量用两个选定的基表示出来,并通过基的过渡矩阵相联系来求证。此略
\end{proof}

\begin{definition}[两个张量的标量积]
两个张量$\mathbf{A},\mathbf{B}\in\mathcal{L}\left(\mathcal{V}\right)$的内积或标量积,记为$\mathbf{A}:\mathbf{B}$,是一个标量。这一运算是一个由$\mathcal{L}\left(\mathcal{V}\right)$到$\mathbb{R}$的映射,定义为:$\mathbf{A}:\mathbf{B}=\mathrm{tr}\left(\mathbf{A}^\intercal\mathbf{B}\right)$。
\end{definition}

易验证张量的标量积满足内积一般定义,故带有张量的标量积规定的空间$\mathcal{L}\left(\mathcal{V}\right)$是一个内积空间。

\begin{theorem}
设$\mathcal{V}$是$\mathbb{R}$上的向量空间,$\mathbf{T}\in\mathcal{L}\left(\mathcal{V}\right)$是$\mathcal{V}$上的一个张量。则$\mathbf{T}$在$\mathcal{V}$的任意基下的表示矩阵的行列式都相等。
\end{theorem}
\begin{proof}
设$A$、$B$是$\mathbf{T}$在$\mathcal{V}$的两组基下的表示矩阵。$P$是这两组基的过渡矩阵,则有$A=P^{-1}BP$。故$\mathrm{det}\left(A\right)=\mathrm{det}\left(P^{-1}BP\right)=\mathrm{det}\left(P^{-1}\right)\mathrm{det}\left(B\right)\mathrm{det}\left(P\right)=\mathrm{det}\left(B\right)$
\end{proof}

由这一性质,我们可以直接称张量在任一基下的矩阵的行列式为这个张量的行列式。

\begin{definition}
设$\mathcal{V}$是$\mathbb{R}$上的向量空间,$\mathbf{T}\in\mathcal{L}\left(\mathcal{V}\right)$是$\mathcal{V}$上的一个张量。则$\mathbf{T}$的行列式$\mathrm{det}\mathbf{T}$是其在$\mathcal{V}$的任意基下的表示矩阵的行列。
\end{definition}
\end{document}