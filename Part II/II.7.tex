\documentclass[main.tex]{subfiles}
% 线性算符的特征值分解
\begin{document}
如果张量$\mathbf{T}:\mathcal{V}\rightarrow\mathcal{V}$在$\mathcal{V}$的某组基$B=\left\{\mathbf{a}_i\right\}_{i=1}^n$下的表示矩阵是对角矩阵
\[
\left(\mathbf{T}\right)=\left(\begin{array}{ccc}
\lambda_1&&\\
&\ddots&\\
&&\lambda_n
\end{array}
\right)\equiv D
\]
那么就有$\mathbf{Ta}_i=\lambda_i\mathbf{a}_i,i=1,\cdots,n$。$\mathbf{T}$的值域就能被$\lambda_i\neq 0$对应的那些$\mathbf{a}_i$线性生成,$\mathbf{T}$的核空间则被$\lambda_i=0$对应的那些$\mathbf{a}_i$线性生成。因此,如果我们找得到这样的基$B$和对角矩阵$D$,就能得知关于线性变换$\mathbf{T}$的很多性质。

\begin{definition}[特征值、特征向量、特征空间]
设$\mathcal{V}$是数域$\mathbb{F}$上的向量空间,$\mathbf{T}\in\mathcal{L}\left(\mathcal{V}\right)$是作用在$\mathcal{V}$上的线性算符。如果存在$\mathbf{a}\neq\mathbf{0}\in\mathcal{V},c\in\mathbb{F}$满足$\mathbf{Ta}=c\mathbf{a}$,则称$c$是$\mathbf{T}$的一个特征值,$\mathbf{a}$是$c$所对应的一个特征向量。$\mathbf{T}$的所有与$c$对应的特征向量组成$\mathbf{T}$的与$c$对应的特征空间。
\end{definition}

\begin{theorem}
设$\mathbf{V}$是数域$\mathbb{F}$上的有限维向量空间,$\mathbf{T}\in\mathcal{L}\left(\mathcal{V}\right)$是作用在$\mathcal{V}$上的线性算符,以下命题相互等价:
\begin{itemize}
    \item $c\in\mathbb{F}$是$\mathbf{T}$的一个特征值
    \item 线性算符$\left(\mathbf{T}-c\mathbf{I}\right)$不可逆
    \item $\mathrm{det}\left(\mathbf{T}-c\mathbf{I}\right)=0$
\end{itemize}
\end{theorem}
\begin{proof}
证明1$\Leftrightarrow$2。设$\mathbf{a}\in\mathcal{V}$满足$\mathbf{Ta}=c\mathbf{a}\Leftrightarrow\left(\mathbf{T}-c\mathbf{I}\right)\mathbf{a}=\mathbf{0}$。满足条件的$\mathbf{a}$至少有$\mathbf{a}=\mathbf{0}$。设所有满足此条件的向量组成的空间是$C$,则$\left(\mathbf{T}-c\mathbf{I}\right)\mathbf{a}=\mathbf{0}\Leftrightarrow C$是$\left(\mathbf{T}-c\mathbf{I}\right)$的核空间。$C$不是只含一个零向量$\mathbf{0}\Leftrightarrow \left(\mathbf{T}-c\mathbf{I}\right)$不是双射$\Leftrightarrow$至少存在一个$\mathbf{a}\neq \mathbf{0}\Leftrightarrow c$是$\mathbf{T}$的一个特征值。

证明2$\Leftrightarrow$3。由定理\cite[p.~51]{周胜林2012线性代数}“$n\times n$矩阵$A$可逆当且仅当$\mathrm{det}A\neq 0$”直接得证。
\end{proof}

$\mathrm{det}\left(\mathbf{T}-c\mathbf{I}\right)$是关于$c$的多项式,阶数与$\mathcal{V}$的维数相同。因此$\mathbf{T}$的特征值$c$是这个多项式的根。

\begin{definition}
设$A$是数域$\mathbb{F}$上的$n\times n$矩阵。若$c\in\mathbb{F}$使$\left(A-cI\right)$为不可逆矩阵,则称$c$是矩阵$A$的特征值。多项式$f\left(x\right)\equiv\mathrm{det}\left(xI-A\right)$称矩阵$A$的特征多项式。矩阵$A$的特征值$c$就是$f\left(c\right)=0$的根。
\end{definition}

\begin{definition}[可对角化线性算符]
设$\mathbf{T}$是数域$\mathbb{F}$上的有限维向量空间$\mathcal{V}$的线性算符,若存在$\mathcal{V}$的一组基的基向量恰好都是$\mathbf{T}$的特征向量,则称$\mathbf{T}$是可对角化的线性算符。
\end{definition}

\begin{example}
设矩阵
\[A=\left(\begin{array}{ccc}3&1&-1\\2&2&-1\\2&2&0\end{array}\right)\]
则其特征多项式
\[\left|\begin{array}{ccc}x-3&-1&1\\-2&x-2&1\\-2&-2&x\end{array}\right|=x^3-5x^2+8x-4=\left(x-1\right)\left(x-2\right)^2\]
可知$A$的特征值是$1,2$。若$A$是$\mathbb{R}^3$上的线性变换$\mathbf{T}$在标准基下的矩阵,由
\[A-1I=A-I=\left(\begin{array}{ccc}2&1&-1\\2&1&-1\\2&2&-1\end{array}\right)\]
得$\mathrm{rank}\left(A-I\right)=2$,即$\mathbf{T}-1\mathbf{I}$的零化度为1,由1对应的特征向量组成的子空间是1维的。同理可得由2对应的特征向量组成的子空间也是1维的。易验$\mathbf{a}_1=\left(1,0,2\right),\mathbf{a}_2=\left(1,1,2\right)$分别是1,2对应的特征向量。由于$\left\{\mathbf{a}_1,\mathbf{a}_2\right\}$无法成为$\mathbb{R}^3$的基,故$\mathbf{T}$不可对角化。
\end{example}

由上例可知,线性算符可对角化的条件,也特征空间的维数有关。

\begin{lemma}
设$\mathbf{T}$是数域$\mathbb{F}$上的有限维向量空间$\mathcal{V}$的线性算符,若$f$是一个多项式,则$f\left(\mathbf{T}\right)\mathbf{a}=f\left(c\right)\mathbf{a}\forall\mathbf{a}\in\mathcal{V}$。
\end{lemma}
\begin{proof}
记$\mathbf{T}$的幂为复合线性变换$\mathbf{T}^n=\mathbf{T}\cdots\mathbf{T}$,则线性变换的多项式$f\left(\mathbf{T}\right)$的意义是明确的,且由线性变换的运算法则易证,略。
\end{proof}

\begin{lemma}
设$\mathbf{T}$是数域$\mathbb{F}$上的有限维向量空间$\mathcal{V}$的线性算符,$c_1,\cdots,c_k$是$\mathbf{T}$的两两不同特征值。令$\mathcal{W}_i$是$c_i$对应的特征向量的空间,若$\mathcal{W}=\mathcal{W}_1+\cdots+\mathcal{W}_k$,则$\mathrm{dim}\mathcal{W}=\mathrm{dim}\mathcal{W}_1+\cdots+\mathrm{dim}\mathcal{W}_k$。若$B_i$是$\mathcal{W}_i$的一组基,则$B=\bigcup_{i=1}^nB_i$是$\mathcal{W}$的一组基。
\end{lemma}
\begin{proof}
一般地$\mathrm{dim}\mathcal{W}\leq\mathrm{dim}\mathcal{W}_1+\cdots+\mathrm{dim}\mathcal{W}_k$。要证等号成立,就要找到$\mathbf{b}_i\in\mathcal{W}_i,i=1,\cdots,k$使得$\mathbf{b}_1+\cdots+\mathbf{b}_k=\mathbf{0}$。

令$f$是任一多项式。由上面的引理有
\[
f\left(\mathbf{T}\right)\mathbf{0}=f\left(\mathbf{T}\right)\left(\sum_{i=1}^k\mathbf{b}_i\right)=\sum_{i=1}^kf\left(c_i\right)\mathbf{b}_i\]
我们能够选择多项式$f_1,\cdots,f_k$使得$f_i\left(c_j\right)=\delta_{ij},i,j=1,\cdots,k$,则$\mathbf{0}=f_i\left(\mathbf{T}\right)\mathbf{0}=\sum_{j=1}^k\delta_{ij}\mathbf{b}_j=\mathbf{b}_i,i=1,\cdots,k$,从而找到了满足要求的$\mathbf{b}_i$。

若$B_i$是$\mathcal{W}_i$的一组基,则$B=\bigcup_{i=1}^kB_i$线性生成$\mathcal{W}$,而且$B$是线性无关向量组,因为任一$\mathbf{B}$中向量的线性变换均可整理成$\sum_{i=1}^k\mathbf{b})i$的形式,其中$\mathbf{b_i}$是$B_i$的基向量的线性组合,且$\sum_{i=1}^k\mathbf{b}_i=\mathbf{0}$。
\end{proof}

\begin{theorem}
设$\mathbf{T}$是数域$\mathbb{F}$上的有限维向量空间$\mathcal{V}$的线性算符,$c_1,\cdots,c_k$是$\mathbf{T}$的两两不同特征值,$\mathcal{W}_i$是$\mathbf{T}-c_i\mathbf{I}$的核空间,则以下命题两两等价。
\begin{enumerate}
    \item $\mathbf{T}$可对角化
    \item $\mathbf{T}$的特征多项式是$f\left(x\right)=\left(x-c_1\right)^{d_1}\cdots\left(x-c_k\right)^{d_k}$且$\mathrm{dim}\mathcal{W}_i=d_i,i=1,\cdots,k$
    \item $\mathrm{dim}\mathcal{W}_1+\cdots\mathrm{dim}\mathcal{W}_k=\mathrm{dim}\mathcal{W}$
\end{enumerate}
\end{theorem}
\begin{proof}
由前面的引理,1$\Leftrightarrow$2已得证,且也易有2$\Rightarrow$3。若3成立,由引理有$\mathcal{V}=\sum_{i=1}^k\mathcal{W}_i$,即$\mathbf{T}$的特征向量线性生成$\mathcal{V}$。
\end{proof}

\begin{example}
若$\mathbb{R}^3$上的线性算符$\mathbf{T}$在标准基下的矩阵
\begin{align*}
\left(\mathbf{T}\right)&=\left(\begin{array}{ccc}5&-6&-6\\-1&4&2\\3&-6&-4\end{array}\right)\\
\mathrm{det}\left(x\mathbf{I}-\mathbf{T}\right)&=\left(x-2\right)^2\left(x-1\right),c_1=1,c_2=2\\
A-I&=\left(\begin{array}{ccc}4&-6&6\\-1&4&2\\3&-6&5\end{array}\right),A-2I=\left(\begin{array}{ccc}3&-6&-6\\-1&2&2\\3&-6&-6\end{array}\right)\\
\mathrm{rank}\left(A-I\right)&=2,\mathrm{rank}\left(A-2I\right)=1\\
\mathrm{dim}\mathcal{W}_1&=1,\mathrm{dim}\mathcal{W}_2=2\\
\therefore \mathrm{dim}\mathcal{W}_1+\mathrm{dim}\mathcal{W}_2&=\mathrm{dim}\mathcal{V}
\end{align*}
所以$\mathbf{T}$可对角化。具体地,找到$\mathcal{W}_1,\mathcal{W}_2$的基$B_1=\left\{\left(3,-1,3\right)\right\},B_2=\left\{\left(2,1,0\right),\left(2,0,1\right)\right\}$,则$\mathcal{V}$的基$B=B_1\cup B_2$。有基向作列形成的矩阵
\[
P=\left(\begin{array}{ccc}
3&2&2\\-1&1&0\\3&0&2\end{array}\right)
\]
可将$A=\left(\mathbf{T}\right)$对角化,对角元素是$\mathbf{T}$的特征值:
\[P^{-1}AP\equiv D=\left(\begin{array}{ccc}1&&\\&2&\\&&2\end{array}\right)\]
\end{example}

\begin{definition}[线性算符的不变量]
设$\mathbf{T}$是作用于数域$\mathbb{F}$上向量空间$\mathcal{V}$的线性算符。$\mathbf{T}$的不变量是$\mathbf{T}$在任一基$B\subset\mathcal{V}$下的矩阵的特征多项式的系数。
\end{definition}

\begin{theorem}
设$A$是一个$n\times n$矩阵,$f\left(\lambda\right)=\mathrm{det}\left(A-\lambda I\right)$是其特征多项式,则$f\left(\lambda\right)$是一个$n$阶多项式,且具有如下形式
\[
f\left(\lambda\right)=\left(-1\right)^n\lambda^n+\left(-1\right)^{n-1}\mathrm{tr}\mathbf{A}\lambda^{n-1}+\cdots+\mathrm{det}A
\]
\end{theorem}
\begin{proof}
作为练习\footnote{参考\cite[\S,p.~118]{周胜林2012线性代数}性质1.2中的式(1.2)的推导过程。更一般地可参考Cayley-Hamilton Theorem。}。
\end{proof}

特别地,当$n=3$时,
\[
f\left(\lambda\right)=
-\lambda^3+\mathrm{tr}\left(A\right)\lambda^2-\frac{1}{2}\left[\mathrm{tr}^2\left(A\right)-\mathrm{tr}\left(A^2\right)\right]\lambda+\mathrm{det}\left(A\right)\]

我们把
\begin{align*}
I_\mathbf{T}&=\mathrm{tr}\mathbf{T}\\
II_\mathbf{T}&=\frac{1}{2}\left(\mathrm{tr}^2\mathbf{T}-\mathrm{tr}\mathbf{T}^2\right)\\
III_\mathbf{T}&=\mathrm{det}\mathbf{T}
\end{align*}
称为张量$\mathbf{T}$的第一、第二和第三主不变量(principal invariants)。我们经常还使用
\begin{align*}
    J_1&=I_\mathbf{T}\\
    J_2&=I_\mathbf{T}^2-2II_\mathbf{T}\\
    J_3&=I_\mathbf{T}^3-3I_\mathbf{T}II_\mathbf{T}+3III_\mathbf{T}
\end{align*}
也被称为主不变量(main invariants),故在使用时要注意以具体定义为准。
\end{document}