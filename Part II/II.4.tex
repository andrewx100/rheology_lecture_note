\documentclass[main.tex]{subfiles}
% 线性变换
\begin{document}
在本节我们考虑一个向量空间到另一个向量空间的映射。

例:
\begin{itemize}
    \item $f:\mathbb{R}\rightarrow\mathbb{R}:f\left(x\right)=x^2$
    \item $g:\mathbb{R}^2\rightarrow\mathbb{R}:g\left(x,y\right)=x^2+y^2-4$
    \item $F:\mathbb{R}^2\rightarrow\mathbb{C}:F\left(x,y\right)=U\left(x,y\right)+iV\left(x,y\right),U,V:\mathbb{R}^2\rightarrow\mathbb{R}$
    \item $T:\mathbb{R}\rightarrow\mathbb{R}^2:T\left(t\right)=\left(t+3,2t-5\right)$
\end{itemize}

一般地,建立了映射的两个向量空间是在同一数域上的,但向量空间之间的映射不一定保留其运算规则。即$f\left(\alpha\mathbf{a}+\beta\mathbf{b}\right)\neq\alpha f\left(\mathbf{a}\right)+\beta f\left(\mathbf{b}\right)$。

\begin{definition}[线性变换]\label{def:II.4.1}
$\mathcal{V}$和$\mathcal{W}$是$\mathbb{F}$上的向量空间。如果从$\mathcal{V}$到$\mathcal{W}$的映射$T:\mathcal{V}\rightarrow\mathcal{W}$满足
\[T\left(\alpha\mathbf{a}+\beta\mathbf{b}\right)=\alpha T\left(\mathbf{a}\right)+\beta T\left(\mathbf{b}\right)\]
则称$T$是从$\mathcal{V}$到$\mathcal{W}$的线性变换。常写成算符的形式:$T\left(\mathbf{a}\right)=\mathbf{Ta}$。如果两个线性变换$\mathbf{T}:\mathcal{V}\rightarrow\mathcal{W}$和$\mathbf{U}:\mathcal{V}\rightarrow\mathcal{W}$满足$\mathbf{Ta}=\mathbf{Ua}\forall\mathbf{a}\in\mathcal{V}$则称这两个线性变换相等,$\mathbf{U}=\mathbf{T}$。零变换$\mathbf{T}_0:\mathcal{V}\rightarrow\mathcal{W}$定义为$\mathbf{T}_0\mathbf{a}=\mathbf{0}_\mathcal{W}\forall\mathbf{a}\in\mathcal{V}$,其中$\mathbf{0}_\mathcal{W}\in\mathcal{W}$表示$\mathcal{W}$中的零向量\footnote{此处要注意区分不同向量空间中的零向量。}。
\end{definition}

易验两个向量空间之间的恒等映射是$\mathcal{V}$到其自身的线性变换$\mathbf{I}_\mathcal{V}:\mathcal{V}\rightarrow\mathcal{V},\mathbf{I}_\mathcal{V}\mathbf{a}=\mathbf{a}\forall\mathbf{a}\in\mathcal{V}$又可称为恒等线性变换。

\begin{theorem}\label{thm:II.4.1}
设$\mathcal{V}$、$\mathcal{W}$是$\mathbb{F}$上的向量空间,$\mathbf{T}$、$\mathbf{U}$是从$\mathcal{V}$到$\mathcal{W}$的线性变换。若定义
\[\left(\mathbf{T}+\mathbf{U}\right)\mathbf{a}=\mathbf{Ta}+\mathbf{Ua},\forall\mathbf{a}\in\mathcal{V}\]
\[\left(\alpha\mathbf{T}\right)\mathbf{a}=\alpha\left(\mathbf{Ta}\right),\forall\alpha\in\mathbb{F}\]
则$\left(\mathbf{T}+\mathbf{U}\right)$和$\alpha\mathbf{T}$也是从$\mathcal{V}$到$\mathcal{W}$的线性变换。
\end{theorem}
\begin{proof}
分别用$\mathbf{T}+\mathbf{U}$和$\alpha\mathbf{T}$作用于向量$\beta\mathbf{a}+\gamma\mathbf{b},\alpha,\beta\in\mathbb{F},\mathbf{a},\mathbf{b}\in\mathcal{V}$,使用向量空间的定义\ref{def:II.2.1}、线性变换的定义\ref{def:II.4.1}和本命题中的运算定义证明。略。
\end{proof}

\begin{corollary}
给定数域$\mathbb{F}$上的两个向量空间$\mathcal{V},\mathcal{W}$,所有由$\mathcal{V}$到$\mathcal{W}$的线性变换的集合组成一个向量空间,记为$\mathcal{L}\left(\mathcal{V},\mathcal{W}\right)$,零变换是该向量空间的零向量。
\end{corollary}
\begin{proof}
前一句由定理\ref{thm:II.4.1}易证。设$\mathbf{T}_0$是由$\mathcal{V}$到$\mathcal{W}$的零变换,对任一$\mathbf{T}\in\mathcal{L}\left(\mathcal{V},\mathcal{W}\right)$和$\mathbf{a}\in\mathcal{V}$,$\left(\mathbf{T}_0+\mathbf{T}\right)\mathbf{a}=\mathbf{T}_0\mathbf{a}+\mathbf{Ta}=\mathbf{0}_\mathcal{W}+\mathbf{Ta}=\mathbf{Ta}$,即$\mathbf{T}_0+\mathbf{T}=\mathbf{T}\forall\mathbf{T}\in\mathcal{L}\left(\mathcal{V},\mathcal{W}\right)$。
\end{proof}

线性变换的空间$\mathcal{L}\left(\mathcal{V},\mathcal{W}\right)$作为一个向量空间,拥有一切向量空间的一般性质。我们在这里考察$\mathcal{L}\left(\mathcal{V},\mathcal{W}\right)$的基和维数。

\begin{theorem}\label{thm:II.4.2}
设$\mathcal{V}_N$是数域$\mathbb{F}$上的$N$维向量空间,$B_\mathcal{V}=\left\{\mathbf{e}_i\right\}_{i=1}^N$是$\mathcal{V}_N$的一组基。$\mathcal{W}$是同数域上的另一向量空间,对每组$N$个不同向量$\left\{\mathbf{b}_i\right\}_{i=1}^N\in\mathcal{W}$,有且只有一个线性变换$\mathbf{T}:\mathcal{V}_N\rightarrow\mathcal{W}$满足$\mathbf{Te}_i=\mathbf{b}_i,i=1,\cdots,N$。
\end{theorem}
\begin{proof}
存在性的证明,只需找出这一线性变换。

任一$\mathbf{a}\in\mathcal{V}_N$可用基$B_\mathcal{V}$表示成
\[\mathbf{a}=\sum_{i=1}^N\alpha_i\mathbf{e}_i,\alpha_i\in\mathbb{F},i=1,\cdots,N\]
定义映射$\mathbf{T}:\mathcal{V}_N\rightarrow\mathcal{W},\mathbf{Ta}\equiv\sum_{i=1}^N\alpha_i\mathbf{b}_i$,则有$\mathbf{Te}_i=\mathbf{b}_i,i=1,\cdots,N$。我们还需验证$\mathbf{T}$是否线性变换,按定义\ref{def:II.4.1}只需验证$T\left(\alpha\mathbf{a}+\beta\mathbf{b}\right)=\alpha\left(\mathbf{Ta}\right)+\beta\left(\mathbf{Tb}\right)$即可。此略。

唯一性的证明,设另有一线性变换$\mathbf{U}:\mathcal{V}_N\rightarrow\mathcal{W}$也满足$\mathbf{Ue}_i=\mathbf{b}_i$,然后只需显示$\mathbf{Ua}=\mathbf{Ta}$即$\mathbf{U}=\mathbf{T}$即可。此略。
\end{proof}

\begin{theorem}\label{thm:II.4.3}
若$\mathcal{V}_N,\mathcal{W}_M$分别为数域$\mathbb{F}$上的$N,M$维向量空间,则$\mathcal{L}\left(\mathcal{V}_N,\mathcal{V}_M\right)$的维数是$N\times M$。
\end{theorem}
\begin{proof}
证明的过程,相当于找出$\mathcal{L}\left(\mathcal{V}_N,\mathcal{W}_M\right)$的基。

设$B_\mathcal{V}=\left\{\mathbf{e}_i\right\}_{i=1}^N,B_\mathcal{W}=\left\{\mathbf{f}_j\right\}_{j=1}^M$分别为$\mathcal{V}_N,\mathcal{W}_M$的基。对于每一对整数$\left(p,q\right),1\leq p\leq N,1\leq q\leq M$,定义一个线性变换$\mathbf{E}^{pq}:\mathcal{V}_N\rightarrow\mathcal{W}_M$\footnote{这一线性变换与例\ref{exp:II.3.2}中的矩阵很类似,可以比较理解。},
\[
\mathbf{E}^{pq}\mathbf{e}_i=\left\{\begin{array}{cc}
    \mathbf{0}_\mathcal{W},&i\neq p  \\
     \mathbf{f}_q,&i=p 
\end{array}\right.
\]
则对于每对$\left(p,q\right)$,$\mathbf{E}^{pq}\in\mathcal{L}\left(\mathcal{V}_N,\mathcal{W}_M\right)$均唯一存在。给定一线性变换$\mathbf{T}\in\mathcal{L}\left(\mathcal{V}_N,\mathcal{W}_M\right)$,且令
\[\mathbf{Te}_i=\sum_{q=1}^MA_{qi}\mathbf{f}_q\]
由定理\ref{thm:II.4.2},对$\left\{\mathbf{f}_j\right\}_{j=1}^M$,$\mathbf{T}$是维一的。且上式中的$A_{qi}$就是向量$\mathbf{Te}_i$的坐标。

定义$\mathbf{U}\in\mathcal{L}\left(\mathcal{V}_N,\mathcal{W}_M\right)$,$\mathbf{Ue}_i=\sum_p\sum_qA_{qp}\mathbf{E}^{pq}\mathbf{e}_i$,则
\begin{align*}
    \mathbf{Ue}_i&=\sum_p\sum_qA_{qp}\mathbf{E}^{pq}\mathbf{e}_i\\
    &=\sum_p\sum_qA_{qp}\delta_{ip}\mathbf{f}_q\\
    &=\sum_qA_{qi}\mathbf{f}_q\\
    &=\mathbf{Te}_i
\end{align*}
则可见$\mathbf{U}=\mathbf{T}$,同时有
\[\mathbf{T}=\sum_{p=1}^N\sum_{q=1}^MA_{qp}\mathbf{E}^{pq}\]
即$\mathbf{T}$可由$\left\{\mathbf{E}^{pq}\right\}$线性表出,换句话说$\mathcal{L}\left(\mathcal{V}_N,\mathcal{W}_M\right)$是$\left\{\mathbf{E}^{pq}\right\}$的线性生成空间。

现在只需再证$\left\{\mathbf{E}^{pq}\right\}$是线性无关的,它们就是$\mathcal{L}\left(\mathcal{V}_N,\mathcal{W}_M\right)$的基了。如果$\mathbf{T}$是零变换,由于$\left\{\mathbf{f}_j\right\}_{j=1}^M$是线性无关的,
\[\mathbf{Te}_i=\sum_q A_{qi}\mathbf{f}_q=\mathbf{0}_\mathcal{W}\Leftrightarrow A_{qi}=0\forall q=1,\cdots,M,i=1,\cdots,N\]
即$\left\{\mathbf{E}^{pq}\right\}$线性无关。故$\left\{\mathbf{E}^{pq}\right\}$是$\mathcal{L}\left(\mathcal{V}_N,\mathcal{W}_M\right)$的一组基,$\mathcal{L}\left(\mathcal{V}_N,\mathcal{W}_M\right)$的维数就是$N\times M$。
\end{proof}

\begin{theorem}
设$\mathcal{V}$、$\mathcal{W}$、$\mathcal{Z}$是$\mathbb{F}$上的向量空间,$\mathbf{T}:\mathcal{V}\rightarrow\mathcal{W}$、$\mathbf{U}:\mathcal{W}\rightarrow\mathcal{Z}$是线性变换,若定义:
\[\mathbf{UT}:\mathcal{V}\rightarrow\mathcal{Z}:\left(\mathbf{UT}\right)\mathbf{a}=\mathbf{U}\left(\mathbf{Ta}\right)\]
则$\mathbf{UT}$也是一个线性变换,称为$\mathbf{U}$和$\mathbf{T}$的复合变换
\label{theo:composed_transform}
\end{theorem}
\begin{proof}
用$\mathbf{UT}$作用于任一向量$\gamma\mathbf{a}+\mathbf{b},\gamma\in\mathbb{F},\mathbf{a},\mathbf{b}\in\mathcal{V}$,证明其等于$\gamma\left(\mathbf{UT}\right)\mathbf{a}+\left(\mathbf{UT}\right)\mathbf{b}$即可。略。
\end{proof}

现在我们考察线性变换作为映射的性质,主要关注的是双射的情况。为此我们需要引入同态映射与同构映射的概念,还要引入零空间、零化度、秩等概念。

\begin{definition}[同态映射]
如果集合$A,B$中定义了一种相同的$k$元操作$\mu$,使得映射$f:A\rightarrow B$满足
\[f\left(\mu\left(\alpha_1,\cdots,\alpha_k\right)\right)=\mu\left(f\left(\alpha_1\right),\cdots,f\left(\alpha_k\right)\right),\forall\alpha_1,\cdots,\alpha_k\in A\]
且$A,B$是同一种代数结构,则称$f$是这种代数结构的一个同态映射。同态映射$f:A\rightarrow A$称为自同态映射。如果$A$和$B$之间存在一个同态映射,则称$A$和$B$是同态的。
\end{definition}

由定义可知,线性变换是两向量空间之间的一种同态映射。

\begin{definition}[同构映射]
如果同态映射$f:A\rightarrow B$是双射,则称$f$是该代数结构的一个同构映射。同构映射$f:A\rightarrow A$称为自同构映射。如果$A$和$B$之间存在一个同构映射,则称$A$和$B$是同构的。
\end{definition}

例:$\mathbb{C}$和$\mathbb{R}^2$都是$\mathbb{R}$上的二维向量空间。它们之间存在同构映射:$\mathbf{T}:\mathbb{C}\rightarrow\mathbb{R}^2:\forall \alpha=\alpha_1+i\alpha_2\in\mathbb{C},\left(a_1,a_2\right)\in\mathbb{R}^2,\mathbf{T}\alpha=\left(\alpha_1,\alpha_2\right),\mathbf{T}^{-1}\left(a_1,a_2\right)=a_1+ia_2$。

\begin{theorem}
设线性变换$\mathbf{T}:\mathcal{V}\rightarrow\mathcal{W}$是数域$\mathbb{F}$上的两个向量空间$\mathcal{V},\mathcal{W}$之间的一个同态映射,则
\begin{enumerate}
    \item 给定$\mathcal{V}$和$\mathcal{W}$的零向量$\mathbf{0}_\mathcal{V}\in\mathcal{V},\mathbf{0}_\mathcal{W}\in\mathcal{W}$,有$\mathbf{T0}_\mathcal{V}=\mathbf{0}_\mathcal{W}$
    \item $\mathbf{T}\left(-\mathbf{a}\right)=-\mathbf{Ta}\forall\mathbf{a}\in\mathcal{V}$
    \item 若$\left\{\mathbf{a}_i\right\}_{i=1}^n\in\mathcal{V}$线性相关,则$\left\{\mathbf{Ta}_i\right\}_{i=1}^n$线性相关。
\end{enumerate}
\end{theorem}

\begin{theorem}
设线性变换$\mathbf{T}:\mathcal{V}\rightarrow\mathcal{W}$是数域$\mathbb{F}$上的两个向量空间$\mathcal{V},\mathcal{W}$之间的一个同构映射,则
\begin{enumerate}
    \item 若$\left\{\mathbf{a}_i\right\}_{i=1}^n\in\mathcal{V}$线性无关,则$\left\{\mathbf{Ta}_i\right\}_{i=1}^n$线性无关
    \item 若$B_\mathcal{V}=\left\{\mathbf{e}_i\right\}$是$\mathcal{V}$的一组基,则$\mathbf{T}B_\mathcal{V}\equiv\left\{\mathbf{Te}_i\right\}_{i=1}^n$是$\mathcal{W}$的一组基
    \item 若$\mathcal{V},\mathcal{W}$是有限维向量空间,则$\mathrm{dim}\mathcal{V}=\mathrm{dim}\mathcal{W}$
\end{enumerate}
\end{theorem}

以上两定理的证明留作练习。

\begin{definition}[零空间、零化度、秩]\label{def:II.4.4}
设$\mathcal{V}$和$\mathcal{W}$是数域$\mathbb{F}$上的向量空间。线性变换$\mathbf{T}:\mathcal{V}\rightarrow\mathcal{W}$的零空间是所有满足$\mathbf{Ta}=\mathbf{0}_\mathcal{W}$的向量$\mathbf{a}$的集合。如果$\mathcal{V}$是有限维向量空间,则$\mathbf{T}$的秩是$\mathbf{T}$的值域的维数,$\mathbf{T}$的零化度是$\mathbf{T}$的零空间的维度。
\end{definition}

\begin{theorem}\label{thm:II.4.7}
设$\mathcal{V}$和$\mathcal{W}$是数域$\mathbb{F}$上的向量空间,其中$\mathcal{V}$是有限维向量空间,则对线性变换$\mathbf{T}:\mathcal{V}\rightarrow\mathcal{W}$有
\[
\mathrm{rank}\mathbf{T}+\mathrm{nullity}\mathbf{T}=\mathrm{dim}\mathcal{V}
\]
\end{theorem}
\begin{proof}
设$\mathrm{dim}\mathcal{V}=n$,$\mathcal{N}$是$\mathbf{T}$的零空间,$\mathrm{nullity}\mathbf{T}=\mathrm{dim}\mathcal{N}=k$,$\left\{\mathbf{a}_1,\cdots,\mathbf{a}_k\right\}$是$\mathcal{N}$的一组基。则可在$\mathcal{V}$中继续找到$\left\{\mathbf{a}_{k+1},\cdots,\mathbf{a}_{n}\right\}$与$\left\{\mathbf{a}_1,\cdots,\mathbf{a}_k\right\}$合并成线性无关向量组,从而成为$\mathcal{V}$的基。

由于$\left\{\mathbf{Ta}_1,\cdots,\mathbf{Ta}_n\right\}$线性生成$\mathcal{W}$(由线性变换性质易证),其中由于$\left\{\mathbf{a}_1,\cdots,\mathbf{a}_k\right\}\in\mathcal{N}$,故有$\mathbf{Ta}_j=\mathbf{0}_\mathcal{W}\forall j\leq k$,所以实际上仅$\left\{\mathbf{Ta}_{k+1},\cdots,\mathbf{Ta}_n\right\}$就线性生成$\mathcal{W}$了。我们进一步验证它们是线性无关的。设标量$\gamma_i$满足$\sum_{i=k+1}^n\gamma_i\mathbf{Ta}_i=\mathbf{0}_\mathcal{W}$,即
\[\mathbf{T}\left(\sum_{i=k+1}^n\gamma_i\mathbf{a}_i\right)=\mathbf{0}_\mathcal{W}\]
即$\sum_{i=k+1}^n\gamma_i\mathbf{a}_i\equiv\mathbf{a}\in\mathcal{N}$。向量$\mathbf{a}$在$\mathcal{N}$的基$\left\{\mathbf{a}_1,\cdots,\mathbf{a}_k\right\}$下表示为$\mathbf{a}=\sum_{i=1}^k\alpha_i\mathbf{a}i$。故
\[
\mathbf{a}=\sum_{i=1}^k\alpha_i\mathbf{a}_i=\sum_{j=k+1}^n\gamma_j\mathbf{a}_j
\]
由于$\left\{\mathbf{a}_1,\cdots,\mathbf{a}_n\right\}$是线性无关的,故有且只有$\alpha_1=\cdots=\alpha_k=\gamma_{k+1}=\cdots=\gamma_n=0$,即$\left\{\mathbf{Ta}_{k+1},\cdots,\mathbf{Ta}_n\right\}$是线性无关的。因此$\left\{\mathbf{Ta}_{k+1},\cdots,\mathbf{Ta}_n\right\}$是$\mathcal{W}$的基,即$\mathcal{W}$的维数是$n-k$,由定义\ref{def:II.4.4}即$\mathrm{rank}\mathbf{T}=n-k$。
\end{proof}

\begin{theorem}\label{thm:II.4.8}
数域$\mathbb{F}$上的两个向量空间同构当且仅当它们维数相等。
\end{theorem}
\begin{proof}
首先证明两个维数相等的向量空间之间必存在一个同构线性变换。设$\mathcal{V}_N$、$\mathcal{W}_N$是数域$\mathbb{F}$上的两个$N$维向量空间。给定$\mathcal{V}_N$的一组基$B_\mathcal{V}=\left\{\mathbf{e}_i\right\}_{i=1}^N$,则向量$\mathbf{a}\in\mathcal{V}_N$可表示为$\mathbf{a}=\sum_{i=1}^N\alpha_i\mathbf{e}_i$,故$\mathbf{a}=\mathbf{0}_\mathcal{V}\Leftrightarrow\alpha_1=\cdots=\alpha_N=0$。任一线性变换$\mathbf{T}:\mathcal{V}_N\rightarrow\mathcal{W}_N$满足
\[\mathbf{Ta}=\mathbf{T}\sum_{i=1}^N\alpha_i\mathbf{e}_i=\sum_{i=1}^N\alpha_i\mathbf{Te}_i\]

我们先考察$\left\{\mathbf{Te}_i\right\}_{i=1}^N$是否线性无关。设$\left\{\mathbf{Te}_i\right\}_{i=1}^N$线性相关,即存在$\left\{\beta_i\right\}_{i=1}^N\subseteq\mathbb{F}$使得$\sum_{i=1}^N\beta_i\mathbf{Te}_i=\mathbf{0}_\mathcal{W}$且至少有一个$\beta_j\neq0$。若是如此,$\mathrm{dim}\mathcal{W}_N>N$,与已知条件矛盾,故$\left\{\mathbf{Te}_i\right\}_{i=1}^N$线性无关。

然后再考察$\left\{\mathbf{Te}_i\right\}_{i=1}^N$是否线性生成$\mathcal{W}_N$。设$\mathcal{W}^\prime$是$\left\{\mathbf{Te}_i\right\}_{i=1}^N$的线性生成空间且$\mathcal{W}^\prime\subset\mathcal{W}_N$(是真子集)。由于已证$\left\{\mathbf{Te}_i\right\}_{i=1}^N$线性无关,则存在$m>0$个线性无关向量$\left\{\mathbf{b}_i\right\}_{i=1}^m$使得$\left\{\mathbf{Te}_i\right\}_{i=1}^N\cup\left\{\mathbf{b}_i\right\}_{i=1}^m$是$\mathcal{W}_N$的一组基,即$\mathrm{dim}\mathcal{W}_N=N+m>N$,这又与已知条件矛盾,故$\mathcal{W}^\prime=\mathcal{W}_N$,即$\left\{\mathbf{Te}_i\right\}_{i=1}^N$线性生成$\mathcal{W}_N$

所以$\left\{\mathbf{Te}_i\right\}_{i=1}^N$是$\mathcal{W}_N$的一组基。

有了这一结论,我们得以考察$\mathbf{T}$是否双射。作为线性变换的$\mathbf{T}$首先是满射。对任一$\mathbf{c}\in\mathcal{W}_N$,可由$\left\{\mathbf{Te}_i\right\}_{i=1}^N$线性表出,即
\[\mathbf{c}=\sum_{i=1}^N\gamma_i\left(\mathbf{Te}_i\right)\equiv\mathbf{T}\left(\sum_{i=1}^N\gamma_i\mathbf{e}_i\right)=\mathbf{Ta}^\prime,\mathbf{a}^\prime\in\mathbf{V}_N\]
即$\forall\mathbf{c}\in\mathcal{W}_N,\mathbf{c}\in\mathrm{ran}\mathbf{T}$,故$\mathbf{T}$又是单射。总之$\mathbf{T}$是双射,$\mathcal{V}_N$与$\mathcal{W}_N$同构。
\end{proof}

\begin{corollary}
任一$\mathbb{F}$上的$N$维向量空间均与$\mathbb{F}^N$同构
\end{corollary}

因此我们可以代表性地通过讨论$\mathbb{R}^n$这一个向量空间的性质来了解所有$n$维向量空间的性质。

同构线性变换是双射,由定理\ref{thm:II.1.7}必存在唯一逆映射。现在,我们可进一步讨论线性变换的逆映射。

\begin{theorem}
数域$\mathbb{F}$上的有限维向量空间$\mathcal{V}$和$\mathcal{W}$之间的线性变换$\mathbf{T}:\mathcal{V}\rightarrow\mathcal{W}$可逆,则其逆映射$\mathbf{T}^{-1}$也是线性变换。
\end{theorem}
\begin{proof}
令$\mathbf{b}_1=\mathbf{Ta}_1,\mathbf{b}_2=\mathbf{Ta}_2,\mathbf{a}_1,\mathbf{a}_2\in\mathcal{V},\mathbf{b}_1,\mathbf{b}_2\in\mathcal{W}$,则有
\begin{align*}
    \mathbf{T}\left(\alpha_1\mathbf{a}_1+\alpha_2\mathbf{a}_2\right)&=\alpha_1\mathbf{Ta}_1+\alpha_2\mathbf{Ta}_2\\
    &=\alpha_1\mathbf{b}_1+\alpha_2\mathbf{b}_2\\
    \therefore\mathbf{T}^{-1}\left(\alpha_1\mathbf{b}_1+\alpha_2\mathbf{b}_2\right)&=\alpha_1\mathbf{a}_1+\alpha_2\mathbf{a}_2\\
    &=\alpha_1\mathbf{T}^{-1}\mathbf{b}_1+\alpha_2\mathbf{T}^{-1}\mathbf{b}_2,\forall\alpha_1,\alpha_2\in\mathbb{F}
\end{align*}
即$\mathbf{T}^{-1}$是线性变换。
\end{proof}

易证从一个向量空间$\mathcal{V}$到其自身有且只有一个恒等变换$\mathbf{I}_\mathcal{V}$。恒等变换属于一种特别的线性变换空间,我们将在稍后介绍(定义\ref{def:II.4.5})。在此时,我们可以说一个线性变换$\mathbf{T}:\mathcal{V}\rightarrow\mathcal{W}$的逆变换$\mathbf{T}^{-1}$满足$\mathbf{T}^{-1}\mathbf{T}=\mathbf{I}_\mathcal{V}$,同时$\mathbf{TT}^{-1}=\mathbf{I}_\mathcal{W}$。

接下来我们考虑线性变换的可逆性与其维数的关系。

由线性变换的定义,对$\mathbf{T}:\mathcal{V}\rightarrow\mathcal{W}$有$\mathbf{T}\left(\mathbf{a}-\mathbf{b}\right)=\mathbf{Ta}-\mathbf{Tb},\mathbf{a},\mathbf{b}\in\mathcal{V}$。故$\mathbf{Ta}=\mathbf{Tb}\Leftrightarrow\mathbf{T}\left(\mathbf{a}-\mathbf{b}\right)=\mathbf{0}_\mathcal{W}$。如果$\mathbf{a}-\mathbf{b}=\mathbf{0}_\mathcal{V}$,则可知$\mathbf{T}$满足$\mathbf{Ta}=\mathbf{0}_\mathcal{W}\Leftrightarrow\mathbf{a}=\mathbf{0}_\mathcal{V}$。我们称满足这一条件的线性变换是非奇异的。同时易验证,$\mathbf{T}$是非奇异的当且仅当它是单射。

\begin{theorem}\label{thm:II.4.10}
线性变换$\mathbf{T}:\mathcal{V}\rightarrow\mathcal{W}$将$\mathcal{V}$的每个线性无关向量组映射至$\mathcal{W}$的一个线性无关向量组当且仅当$\mathbf{T}$是非奇异的。
\end{theorem}
\begin{proof}
设$\mathbf{T}$是非奇异的。令$S$是$\mathcal{V}$的一个线性无关向量组,若$\mathbf{a}_1,\cdots,\mathbf{a}_k\in S$。考虑以下线性组合的结论
\begin{align*}
&\gamma_1\mathbf{Ta}_1+\cdots+\gamma_k\mathbf{Ta}_k=\mathbf{0}_\mathcal{W}\\
\Leftrightarrow&\mathbf{T}\left(\gamma_1\mathbf{a}_1+\cdots+\gamma_k\mathbf{a}_k\right)=\mathbf{0}_\mathcal{W}\quad\text{(线性变换的性质)}\\
\Leftrightarrow&\gamma_1\mathbf{a}_1+\cdots+\gamma_k\mathbf{a}_k=\mathbf{0}_\mathcal{V}\quad\text{($\mathbf{T}$是非奇异的)}\\
\Leftrightarrow &\gamma_1=\cdots=\gamma_k=0\quad\text{($\mathbf{a}_1,\cdots,\mathbf{a}_k$线性无关)}
\end{align*}
\end{proof}

由定理\ref{thm:II.4.7}和\ref{thm:II.4.8}可得到如下等价命题。

\begin{theorem}\label{thm:II.4.11}
设$\mathcal{V}$和$\mathcal{W}$是数域$\mathbb{F}$上的有限维向量空间且$\mathrm{dim}\mathcal{V}=\mathrm{dim}\mathcal{W}$。对线性变换$\mathbf{T}:\mathcal{V}\rightarrow\mathcal{W}$,以下命题相互等价:
\begin{enumerate}
    \item $\mathbf{T}$可逆
    \item $\mathbf{T}$是非奇异的
    \item $\mathbf{T}$是满射,即$\mathrm{ran}\mathbf{T}=\mathcal{W}$
\end{enumerate}
\end{theorem}
\begin{proof}
定理\ref{thm:II.4.8}直接给出1$\Leftrightarrow$3。

由定理\ref{thm:II.4.7}有$\mathrm{dim}\mathcal{V}=\mathrm{dim}\mathcal{W}\Leftrightarrow\mathrm{nullity}\mathbf{T}=0\Leftrightarrow\mathbf{T}$是非奇异的。
\end{proof}

我们看到,维数相等的向量空间都至少存在一对可逆同构映射保证这些向量空间的同构性,通过这一对可逆同构映射我们总能把从一个向量空间到另一个维数相同的向量空间的线性变换的研究,转化为对从一个向量空间到其自身的线性变换研究。

\begin{definition}[线性算符]\label{def:II.4.5}
设$\mathcal{V}$是数域$\mathbb{F}$上的向量空间,由$\mathcal{V}$到基自身的线性变换$\mathbf{T}:\mathcal{V}\rightarrow\mathcal{V}$称为线性算符。
\end{definition}

显然,线性算符都可逆。而且线性算符与其逆都同属同一个空间$\mathcal{L}\left(\mathcal{V},\mathcal{V}\right)$(简写为$\mathcal{L}\left(\mathcal{V}\right)$)。对于一般的线性变换,本节开头介绍了其向量代数。现在我们可以通过复合变换的定义,为线性算符引入乘法的代数\footnote{代数(algebra)的定义此略,此处只强调,要成为代数的运算,其中一个必要条件是要具有封闭性,既域上的运算结果还在域内。一般的线性变换的复合操作不满足此条件。}。此时恒等变换记号$\mathbf{I}$无需指明作用空间。

\begin{theorem}\label{thm:II.4.12}
设$\mathcal{V}$是数域$\mathbb{F}$上的线性变换,$\mathbf{U},\mathbf{T}_1,\mathbf{T}_2\in\mathcal{L}\left(\mathcal{V}\right)$,$\gamma\in\mathbb{F}$,则有:
\begin{enumerate}
    \item $\mathbf{IU}=\mathbf{UI}=\mathbf{U}$
    \item $\mathbf{U}\left(\mathbf{T}_1+\mathbf{T}_2\right)=\mathbf{UT}_1+\mathbf{UT}_2$,$\left(\mathbf{T}_1+\mathbf{T}_2\right)\mathbf{U}=\mathbf{T}_1\mathbf{U}+\mathbf{T}_2\mathbf{U}$
    \item $\gamma\mathbf{UT}_1=\left(\gamma\mathbf{U}\right)\mathbf{T}_1=\mathbf{U}\left(\gamma\mathbf{T}_1\right)$
\end{enumerate}
\begin{proof}
第1条由相关定义是易证的。

对任一向量$\mathbf{a}\in\mathcal{V}$,
\begin{align*}
    \left[\mathbf{U}\left(\mathbf{T}_1+\mathbf{T}_2\right)\right]\mathbf{a}&=\mathbf{U}\left[\left(\mathbf{T}_1+\mathbf{T}_2\right)\mathbf{a}\right]\quad\text{(由定理\ref{thm:II.4.1}中的定义)}\\
    &=\mathbf{U}\left(\mathbf{T}_1\mathbf{a}+\mathbf{T}_2\mathbf{a}\right)\quad\text{($\mathbf{U}$是线性变换)}\\
    &=\left(\mathbf{UT}_1\right)\mathbf{a}+\left(\mathbf{UT}_2\right)\mathbf{a}\quad\text{(由复合映射的定义)}
\end{align*}
故由两映射相等的定义,$\mathbf{U}\left(\mathbf{T}_1+\mathbf{T}_2\right)=\mathbf{UT}_1+\mathbf{UT}_2$。类似地,
\begin{align*}
    \left[\left(\mathbf{T}_1+\mathbf{T}_2\right)\mathbf{U}\right]\mathbf{a}&=\left(\mathbf{T}_1+\mathbf{T}_2\right)\left(\mathbf{Ua}\right)\quad\text{(由复合映射的定义)}\\
    &=\mathbf{T}_1\left(\mathbf{Ua}\right)+\mathbf{T}_2\left(\mathbf{Ua}\right)\quad\text{(由定理\ref{thm:II.4.1}中的定义)}\\
    &=\left(\mathbf{T}_1\mathbf{U}\right)\mathbf{a}+\left(\mathbf{T}_2\mathbf{U}\right)\mathbf{a}\quad\text{(由复合映射的定义)}
\end{align*}
故由两映射相等的定义,,$\left(\mathbf{T}_1+\mathbf{T}_2\right)\mathbf{U}=\mathbf{T}_1\mathbf{U}+\mathbf{T}_2\mathbf{U}$,第2条证毕\footnote{我们留意到第2条的第一部分证明没有用到$\mathbf{T}_1$和$\mathbf{T}_2$是线性变换的条件,第二部分的证明连$\mathbf{U}$是线变换的条件都没用。}。第3条证明留作练习。
\end{proof}
\end{theorem}
\end{document}