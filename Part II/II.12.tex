\documentclass[main.tex]{subfiles}
% 函数的微分和导数
\begin{document}
%======================================
\subsection{一元函数的导数与微分(回顾)}
我们在本科的高等数学课上已经学过一元函数的导数与微分。我们复述它们,一是为了明确符号的记法,二是为了与后面介绍的多维的情况作对比。

回顾高阶无穷小定义,简述为:若函数$f\left(x\right)$在点$x_0$处有极限$\lim_{x\to x_0}f\left(x\right)=0$,则称函数$f\left(x\right)$是当$x\to x_0$时的一个无穷小。若函数$f\left(x\right),g\left(x\right)$是当$x\to x_0$时的无穷小,且$\lim_{x\to x_0}f\left(x\right)/g\left(x\right)=0$,则称$f\left(x\right)$是当$x\to x_0$时$g\left(x\right)$的一个高阶无穷小。

\begin{definition}[一元函数的导数]\cite[“定义2.1.1”,p.~70]{华工高数2009上}
设函数$y=f\left(x\right)$在点$x_0$的某个邻域内有定义,当自变量$x$在点$x_0$处取得改变量$\Delta x$,且点$x_0+\Delta x$在上述邻域内时,相应地,函数的改变量为
\[
\Delta y=f\left(x_0+\Delta x\right)-f\left(x_0\right)\text{,}
\]
如果极限
\[
\lim_{\Delta x\to 0}\frac{\Delta y}{\Delta x}=\lim_{\Delta x\to 0}\frac{f\left(x_0+\Delta x\right)-f\left(x_0\right)}{\Delta x}
\]
存在,称函数$f\left(x\right)$在点$x_0$处可导(或存在导数),极限值称为函数在点$x_0$处的导数(或微商),记为
\[f^\prime\left(x_0\right),\left.y^\prime\right|_{x=x_0},\left.\frac{df}{dx}\right|_{x=x_0},\text{或}\left.\frac{dy}{dx}\right|_{x=x_0}\text{,}
\]
即
\[
f^\prime\left(x_0\right)=\lim_{\Delta x\to 0}\frac{f\left(x_0+\Delta x\right)-f\left(x_0\right)}{\Delta x}\text{。}
\]
若极限不存在,称函数$f\left(x\right)$在$x_0$处不可导。
\end{definition}

\begin{definition}[一元函数的微分]\cite[“定义2.5.1”,p.~103]{华工高数2009上}
设函数$y=f\left(x\right)$在点$x_0$的某邻域内有定义,当$x$在点$x_0$处获得增量$\Delta x$时,如果相应的函数增量$\Delta y=f\left(x_0+\Delta x\right)-f\left(x_0\right)$可以表示为
\[\Delta y=A\Delta x+o\left(\Delta x\right)\text{,}
\]
其中,常数$A$与$\Delta x$无关(仅与$x_0$有关),而$o\left(\Delta x\right)$是较$\Delta x\left(\Delta x\to 0\right)$的高阶无穷小,则称函数$y=f\left(x\right)$在点$x_0$处可微。且称$A\Delta x$为函数$y=f\left(x\right)$在点$x_0$处对应于自变量增量$\Delta x$的微分,记作$dy$,即
\[
dy=A\Delta x\text{,}
\]
也常把$A\Delta x\left(A\neq 0\right)$称为函数增量$\Delta y$的线性主部。
\end{definition}

在$\Delta y$的分解式中,由于其值主要取决于线性主部$A\Delta x$,因此当$\Delta x\to 0$时,我们可以写
\[
\Delta y\approx dy=A\Delta x\text{。}
\]

\begin{theorem}\cite[“定理2.5.1”,p.~103]{华工高数2009上}
函数$y=f\left(x\right)$在点$x_0$处可微的充分必要条件是:函数$y=f\left(x\right)$在点$x_0$处可导,且当$y=f\left(x\right)$在点$x_0$处可微时,其微分是
\[
dy=f^\prime\left(x_0\right)\Delta x\text{。}
\]
\end{theorem}
\begin{proof}
略\cite[p.~103]{华工高数2009上}。
\end{proof}

关于符号“d”的意义,这里给出一个比高等数学课本\cite[p.~104]{华工高数2009上}更仔细的说明。

由函数微分的定义可知,函数$f\left(x\right)$在点$x_0$处对应于自变量增量$\Delta x$的微分$dy=f'\left(x_0\right)\Delta x$,实际上是一个由点$x_0$引出的函数,记为$df_{x_0}:\mathbb{R}\rightarrow\mathbb{R},df_{x_0}\left(x\right)=f'\left(x_0\right)x,\forall x\in I$,其中区域$I$是所有满足$x_0+x$在函数可微的$x_0$的邻域的所有实数$x$的集合。

若视$x$本身为一个恒等映射,即$x\left(p\right)=p,\forall p\in\mathbb{R}$,则函数$x$在点$p$处的微分也是一个由点$p$引出的函数,且$dx_p\left(u\right)=x^\prime\left(p\right) u=1\times u,\forall u\in\mathbb{R}$,也是一个恒等映射。同时我们看到,函数$dx_p\left(u\right)$是$u\to 0$的无穷小。

考虑复合映射$f\left(x\left(p\right)\right)$在点$p_0$处的微分,它是由点$p_0$引出的函数,\[d\left(f\circ x\right)_{p_0}\left(u\right)=\left.\frac{d\left(f\circ x\right)}{dp}\right|_{p=p_0}u=f^\prime\left(x(p_0\right))x^\prime\left(p_0\right)u=f^\prime\left(p_0\right)dx_{p_0}\left(u\right)\]
而且函数$d\left(f\circ x\right)_{p_0}\left(u\right)$与$dx_{p_0}\left(u\right)$是$u\to 0$时的同阶无穷小,$f^\prime\left(p_0\right)$就是这两个无穷小的比值,该结论对所有$p_0,u\in\mathbb{R}$都成立,因此若去掉$p_0$和$u$简记:
\[df\left(x\right)=f^\prime dx\]
则
\[f^\prime=\frac{df}{dx}\text{。}\]
因此,当我们把导数当作分数来处理时,实际是在上述的意义上视导数为两个微分作为同阶无穷小的比值。

\begin{definition}[一元向量函数的导数]
设函数$\mathbf{f}:\mathbb{R}\supseteq D\rightarrow\mathbb{R}^n$的极限
\[
\lim_{h\to 0}\frac{\mathbf{f}\left(t+h\right)-\mathbf{f}\left(t\right)}{h}
\]
存在,则称函数$\mathbf{f}\left(x\right)$在$x=t$处可导。该极限是函数$\mathbf{f}\left(x\right)$在$x=t$处的导数,记为
\[
\left.\frac{d\mathbf{f}\left(x\right)}{dx}\right|_{x=t}\equiv\lim_{h\to 0}\frac{\mathbf{f}\left(t+h\right)-\mathbf{f}\left(t\right)}{h}
\]
\end{definition}

由一元标量函数求导的法则可证以下一元向量函数求导法则成立:对任意函数$\mathbf{f}:\mathbb{R}\supseteq D\rightarrow\mathbb{R}^n,\mathbf{g}:D\rightarrow\mathbb{R}^n$,
\begin{itemize}
\item $\frac{d\mathbf{c}}{dt}=0,\mathbf{c}\in\mathbb{R}^n$是常向量
\item $\frac{d}{dt}\left(\alpha\mathbf{f}+\beta\mathbf{g}\right)=\alpha\frac{d}{dt}\mathbf{f}+\beta\frac{d}{dt}\mathbf{g},\forall \alpha,\beta\in\mathbb{R}$
\item $\frac{d}{dt}\left[u\left(t\right)\mathbf{f}\left(t\right)\right]=\frac{du}{dt}\mathbf{f}+u\frac{d}{dt}\mathbf{f},\forall u:D\rightarrow\mathbb{R}$
\item $\frac{d}{dt}\left(\mathbf{f}\cdot\mathbf{g}\right)=\frac{d\mathbf{f}}{dt}\cdot\mathbf{g}+\mathbf{f}\cdot\frac{d\mathbf{g}}{dt}$
\item $\frac{d}{dt}\left(\mathbf{f}\times\mathbf{g}\right)=\frac{d\mathbf{f}}{dt}\times\mathbf{g}+\mathbf{f}\times\frac{d\mathbf{g}}{dt}$
\end{itemize}

\begin{definition}[多元标量值函数的偏导数]\label{def:II.12.4}
给定函数$f:\mathbb{R}^n\supseteq D\rightarrow\mathbb{R}$,若对某一$i\in\left\{1,\cdots,n\right\}$极限
\[
\lim_{t\to0}\frac{f\left(\cdots,x_{i}+h,\cdots\right)-f\left(\cdots,x_{i},\cdots\right)}{t}
\]
当$x_i=x_{i0}$时存在,则称该极限为函数$f\left(\mathbf{x}\right)$对第$i$个变量$x_i$在$x_i=x_{i0}$处的的偏导数(partial derivative),记为
\[\left.\frac{\partial f\left(\mathbf{x}\right)}{\partial x_i}\right|_{x_i=x_{i0}}\equiv\lim_{t\to0}\frac{f\left(\cdots,x_{i}+h,\cdots\right)-f\left(\cdots,x_i,\cdots\right)}{t}
\]
\end{definition}

\begin{theorem}\footnote{即高等数学\cite[p.~16]{华工高数2009下}定理7.2.1向$n$维的推广。}
如果函数$f:\mathbb{R}^n\rightarrow\mathbb{R}$的一个二阶混合偏导数$\frac{\partial^2f}{\partial x_i\partial x_j}$在$\mathbb{R}^2$的一个开子集$S$上处处连续,则二阶混合偏导数$\frac{\partial^2f}{\partial x_j\partial x_i}$在$S$上处处存在,且$\frac{\partial^2f}{\partial x_i\partial x_j}=\frac{\partial^2f}{\partial x_j\partial x_i}$。
\end{theorem}
\begin{proof}
略\footnote{进一步了解:\href{https://en.wikipedia.org/wiki/Symmetry_of_second_derivatives}{Symmetry of second derivatives}。}
\end{proof}

\begin{definition}[向量函数的偏导数]
函数$\mathbf{f}:\mathbb{R}^n\supseteq D\rightarrow\mathbb{R}^m$对第$i$个变量在$x_i=x_{i0}$处的偏导数
\[
\left.\frac{\partial \mathbf{f}\left(\mathbf{x}\right)}{\partial x_i}\right|_{x_i=x_{i0}}\equiv\left(\begin{array}{c}
\left.\frac{\partial f_1\left(\mathbf{x}\right)}{\partial x_i}\right|_{x_i=x_{i0}}\\
\vdots\\
\left.\frac{\partial f_m\left(\mathbf{x}\right)}{\partial x_i}\right|_{x_i=x_{i0}}\end{array}\right)\in\mathbb{R}^m
\]
其中,$\mathbf{x}=\left(x_1,\cdots,x_n\right)^\intercal\in\mathbb{R}^n$。
\end{definition}


%==========================================
\subsection{向量函数的微分和导数}
回顾多元标量函数的全微分的定义\cite[“定义7.3.1”,p.~19]{华工高数2009下}:

\begin{definition}[函数的全微分(多元标量值函数)]\label{def:II.12.6}
若函数$z=f\left(x,y\right)$在点$P_0\left(x_0,y_0\right)$的全增量
\[
\Delta z=f\left(x_0+\Delta x,y_0+\Delta y\right)-f\left(x_0,y_0\right)
\]
可表示为
\[
\Delta z=A \Delta x+ B \Delta y+o\left(\rho\right)
\]
其中$A,B$只与点$\left(x_0,y_0\right)$有关,而与$\Delta x,\Delta y$无关。又$\rho=\sqrt{\left(\Delta x\right)^2+\left(\Delta y\right)^2}$,$o\left(\rho\right)$是当$\rho\rightarrow 0$时$\rho$的高阶无穷小,则称函数$z=f\left(x,y\right)$在点$P_0\left(x_0,y_0\right)$可微分,且把$\Delta z$的线性主部$A\Delta x+B\Delta y$称为函数$z=f\left(x,y\right)$在点$P_0\left(x_0,y_0\right)$的全微分,记作
\[
\left.dz\right|_{x=x_0,y=y_0}=A\Delta x+B\Delta y\text{或}df\left(x_0,y_0\right)=A\Delta x+B\Delta y
\]
如果函数$z=f\left(x,y\right)$在区域$D$内每一点都可微,则称这函数在$D$内可微。
\end{definition}

如果函数$z=f\left(x,y\right)$在点$P_0\left(x_0,y_0\right)$处可微,则有
\begin{align*}
    &f\left(x_0+\Delta x,y_0+\Delta y\right)-f\left(x_0,y_0\right)=A\Delta x+B\Delta y+o\left(\rho\right)\\
    \Leftrightarrow&o\left(\rho\right)=f\left(x_0+\Delta x,y_0+\Delta y\right)-f\left(x_0,y_0\right)-\left(A\Delta x+B\Delta y\right)\\
    \Leftrightarrow&\lim_{\rho\to0}\frac{f\left(x_0+\Delta x,y_0+\Delta y\right)-f\left(x_0,y_0\right)-\left(A\Delta x+B\Delta y\right)}{\sqrt{\Delta x^2+\Delta y^2}}=0\\
    \Leftrightarrow&\lim_{\rho\to0}\frac{\Delta z-dz}{\rho}=0
\end{align*}
其中$\rho=\sqrt{\Delta x^2+\Delta y^2}$。上面的极限式是定义\ref{def:II.12.6}的等价定义式。我们其实可以不引入某个高阶无穷小$o\left(\rho\right)$,直接用使该极限式成立的$A,B$的存在性来定义函数的可微性。下面我们按照把函数微分的定义推广到一般的向量函数。

\begin{definition}[函数的微分(向量函数)]\label{def:II.12.7}
若函数$\mathbf{f}:\mathbb{R}^n\supseteq D\rightarrow\mathbb{R}^m$在其定义域内某点$\mathbf{x}_0\in D$处,存在一个线性变换$\mathbf{L}:\mathbb{R}^n\rightarrow\mathbb{R}^m$使得对任意$\mathbf{x}_0$的邻域$N\left(\mathbf{x}_0\right)$中的点$\mathbf{x}\in N\left(\mathbf{x}_0\right)$,
\[\lim_{\mathbf{x}\rightarrow\mathbf{x}_0}\frac{\mathbf{f}\left(\mathbf{x}\right)-\mathbf{x}\left(\mathbf{x}_0\right)-\mathbf{L}\left(\mathbf{x}-\mathbf{x}_0\right)}{\left\|\mathbf{x}-\mathbf{x}_0\right\|}=\mathbf{0}\]
则称函数$\mathbf{f}\left(\mathbf{x}\right)$在$\mathbf{x}_0$处可微分(differentiable)。记$d\mathbf{f}_{\mathbf{x}_0}=\mathbf{L}\left(\mathbf{x}-\mathbf{x}_0\right)$是函数$\mathbf{f}\left(\mathbf{x}\right)$在$\mathbf{x}_0$处的微分(differential)。如果函数$\mathbf{f}\left(\mathbf{x}\right)$在开集$S\subseteq D$内的每一点上都可微分,则称函数是$S$上的可微函数(differentiable function)。
\end{definition}

我们可以根据向量函数的定义来看出,定义\ref{def:II.12.7}就是定义\ref{def:II.12.6}的推广。延用定义\ref{def:II.12.7}的设定,设$\mathbf{f}=\left(f_1,\cdots,f_m\right)^\intercal,\mathbf{x}=\left(x_1,\cdots,x_n\right)^\intercal,\mathbf{x}_0=\left(x_{01},\cdots,x_{0n}\right)^\intercal$,则函数$\mathbf{f}\left(\mathbf{x}\right)$在$\mathbf{x}_0$处的全增量:
\[
    \mathbf{f}\left(\mathbf{x}\right)-\mathbf{f}\left(\mathbf{x}_0\right)=\left(
    \begin{array}{c}
        f_1\left(x_1,\cdots,x_n\right)-f_1\left(x_{01},\cdots,x_{0n}\right)\\
    \vdots\\
    f_m\left(x_1,\cdots,x_n\right)-f_m\left(x_{01},\cdots,x_{0n}\right)
    \end{array}\right)
\]
可见我们实际考虑的是$m$个$n$元标量值函数分别在点$x_{01},\cdots,x_{0n}$处的全增量。如果在这些点上这些标量值函数分别都可微,则$\mathbf{f}$的每个坐标函数的全增量都可按定义\ref{def:II.12.6}写成
\[
f_i\left(x_1,\cdots,x_n\right)-f_i\left(x_{01},\cdots,x_{0n}\right)=\sum_{j=1}^n L_{ji}\left(x_j-x_{0j}\right)+o\left(\rho\right),i=1,\cdots,m
\]
其中,$\rho=\left(\sum_{j=1}^n\left(x_j-x_{0j}\right)^2\right)^{1/2}=\left\|\mathbf{x}-\mathbf{x}_0\right\|$,$o\left(\rho\right)$是当$\rho\to 0$时$\rho$的高阶无穷小。这等价于如下极限式
\[
\lim_{\rho\to 0}\frac{1}{\left\|\mathbf{x}-\mathbf{x}_0\right\|}\left(\begin{array}{c}
f_1\left(x_1,\cdots,x_n\right)-f_1\left(x_{01},\cdots,x_{0n}\right)-\sum_{j-1}^nL_{1j}\left(x_j-x_{0j}\right)\\
\vdots\\
f_m\left(x_1,\cdots,x_n\right)-f_m\left(x_{01},\cdots,x_{0n}\right)-\sum_{j-1}^nL_{mj}\left(x_j-x_{0j}\right)
\end{array}\right)=\left(\begin{array}{c}0\\\vdots\\0\end{array}\right)
\]
若线性变换$\mathbf{L}$在标准基下的矩阵坐标就是$L_{ij}$,则上式等价于\ref{def:II.12.7}的极限式。

我们接下来考察函数可微与可导的关系,即函数可微的充分和必要条件\footnote{这部分内容可以与本科高等数学课上的相应内容对比学习,特别是关于必要非充份条件和充份非必要条件的例子\cite[“二、全微分存在的条件”,p.~20]{华工高数2009下}。}。首先,以下定理是函数在某点处可微分的必要条件,它同时也给出了$\mathbf{L}$或其坐标矩阵$L_{ij}$的计算方法。

\begin{theorem}\label{thm:II.12.3}
设函数$\mathbf{f}:\mathbb{R}^n\supseteq D\rightarrow\mathbb{R}^m$在点$\mathbf{x}_0\in D$处可微分,即存在线性变换$\mathbf{L}\in\mathcal{L}\left(\mathbb{R}^n,\mathbb{R}^m\right)$满足
\[
\lim_{\Delta\mathbf{x}\to\mathbf{0}}\frac{\mathbf{f}\left(\mathbf{x}_0+\Delta \mathbf{x}\right)-\mathbf{f}\left(\mathbf{x}_0\right)-\mathbf{L}\Delta\mathbf{x}}{\left\|\Delta\mathbf{x}\right\|}=\mathbf{0}
\]
则$\mathbf{f}$的每个坐标函数在$\mathbf{x}_0$处的每个偏导数
\[
\left.\frac{f_i\left(\mathbf{x}\right)}{\partial x_j}\right|_{\mathbf{x}=\mathbf{x}_0},i=1,\cdots,m,j=1,\cdots,n
\]
都存在。若$\left\{\mathbf{\hat{e}}_1,\cdots\mathbf{\hat{e}}_n\right\},\left\{\mathbf{\hat{u}}_1,\cdots,\mathbf{\hat{u}}_m\right\}$分别是$\mathbb{R}^n,\mathbb{R}^m$的标准基,则
\[
\mathbf{L\hat{e}}_j=\sum_{i=1}^m\left(\left.\frac{\partial f_i\left(\mathbf{x}\right)}{\partial x_j}\right|_{\mathbf{x}=\mathbf{x}_0}\right)\mathbf{\hat{u}}_i,j=1,\cdots,n
\]
\end{theorem}
\begin{proof}
见附录。
\end{proof}

注意到$\mathbf{L\hat{e}}_j$其实就是线性变换的第$j$列,故上述定理说明,如果函数在某点处可微分,则其微分的线性变换$\mathbf{L}$就是函数的偏导数所形成的矩阵:
\[
\left(\mathbf{L}\right)=\left(\begin{array}{ccc}
\left.\frac{\partial f_1\left(\mathbf{x}\right)}{\partial x_1}\right|_{\mathbf{x}=\mathbf{x}_0}&\cdots&\left.\frac{\partial f_1\left(\mathbf{x}\right)}{\partial x_n}\right|_{\mathbf{x}=\mathbf{x}_0}\\
\vdots&\ddots&\vdots\\
\left.\frac{\partial f_m\left(\mathbf{x}\right)}{\partial x_1}\right|_{\mathbf{x}=\mathbf{x}_0}&\cdots&\left.\frac{\partial f_m\left(\mathbf{x}\right)}{\partial x_n}\right|_{\mathbf{x}=\mathbf{x}_0}
\end{array}\right)
\]

以下定理解决了函数微分的线性变换$\mathbf{L}$的唯一性。

\begin{theorem}\label{thm:II.12.4}
设函数$\mathbf{f}:\mathbb{R}^n\supseteq D\rightarrow\mathbb{R}^m$在点$\mathbf{x}_0\in D$处可微分,即存在线性变换$\mathbf{L}\in\mathcal{L}\left(\mathbb{R}^n,\mathbb{R}^m\right)$满足
\[
\lim_{\Delta\mathbf{x}\to\mathbf{0}}\frac{\mathbf{f}\left(\mathbf{x}_0+\Delta \mathbf{x}\right)-\mathbf{f}\left(\mathbf{x}_0\right)-\mathbf{L}\Delta\mathbf{x}}{\left\|\Delta\mathbf{x}\right\|}=\mathbf{0}
\]
则$\mathbf{L}$是唯一的。
\end{theorem}
\begin{proof}
见附录。
\end{proof}

定理\ref{thm:II.12.3}和\ref{thm:II.12.4}共同构成了函数可微分的必要非充份条件。也就是说,并非每当函数在某点处的所有偏导数都存在,该函数就一定在该点可微\cite[“例2”,p.~21]{华工高数2009下}。不过,有了唯一性,我们至少可以把函数微分的线性变换$\mathbf{L}$定义为函数的导数,具体地——

\begin{definition}[向量函数的导数]
设函数$\mathbf{f}:\mathbb{R}^n\supseteq D\rightarrow\mathbb{R}^m$在点$\mathbf{x}_0\in D$处可微分,即存在线性变换$\mathbf{L}\in\mathcal{L}\left(\mathbb{R}^n,\mathbb{R}^m\right)$满足
\[
\lim_{\Delta\mathbf{x}\to\mathbf{0}}\frac{\mathbf{f}\left(\mathbf{x}_0+\Delta \mathbf{x}\right)-\mathbf{f}\left(\mathbf{x}_0\right)-\mathbf{L}\Delta\mathbf{x}}{\left\|\Delta\mathbf{x}\right\|}=\mathbf{0}
\]
则称线性变换$\mathbf{L}$是函数$\mathbf{f}\left(\mathbf{x}\right)$在$\mathbf{x}_0$处的导数(derivative),记为
\[\mathbf{L}\equiv\left.\frac{d\mathbf{f}\left(\mathbf{x}\right)}{d\mathbf{x}}\right|_{\mathbf{x}=\mathbf{x}_0}\]
$\mathbf{L}$在标准基下的坐标矩阵称为函数$\mathbf{f}\left(\mathbf{x}\right)$在$\mathbf{x}_0$处的雅可比矩阵(Jacobian matrix),$\mathbf{L}$的行列式$\mathrm{det}\mathbf{L}$称函数$\mathbf{f}\left(\mathbf{x}\right)$在$\mathbf{x}_0$处的雅可比行列式(Jacobian determinant)。
\end{definition}

下面我们给出一个函数在某点处可微的充分条件(即未必一定要满足该条件函数才可微,但满足该条件函数必可微)\cite[“例3”,p.~23]{华工高数2009下}。

\begin{theorem}\label{thm:II.12.5}
若函数$\mathbf{f}:\mathbb{R}^n\supset D\rightarrow\mathbb{R}^m$的定义域$D$是开集,偏微分$\frac{\partial f_i}{\partial x_j},i=1,\cdots,n,j=1,\cdots,m$在$D$内都连续,则$\mathbf{f}$在$D$内均可微分。
\end{theorem}
\begin{proof}
见附录。
\end{proof}

我们看到,函数导数的定义引入了类似微商的记法,但是我们没有定义过“矢量的无穷小”和“矢量的除法”。这里我们作一些说明。若函数$\mathbf{f}:\mathbb{R}^n\rightarrow\mathbb{R}^m$在$D\subset\mathbb{R}^n$上可微分,其在某点$\mathbf{x}_0\in D$处的全微分是由点$\mathbf{x}_0$引出的函数$d\mathbf{f}_{\mathbf{x}_0}:\mathbb{R}^n\rightarrow\mathbb{R}^m,d\mathbf{f}_{\mathbf{x}_0}\left(\mathbf{x}\right)=\mathbf{L}_{\mathbf{x}_0}\mathbf{x},\forall \mathbf{x}\in\mathbf{R}^n$\footnote{为了讲述简洁这里略去了对函数$\mathbf{f}$在一点处可微分与相应的增量与该点邻域的关系问题。},其中$\mathbf{L}_{\mathbf{x}_0}$是函数$\mathbf{f}$在$\mathbf{x}_0$处的导数。但是我们直接看到这个引出的函数就是函数的导数$\mathbf{L}_{\mathbf{x}_0}$本身。若视$\mathbf{x}\in D$为$\mathbb{R}^n$上的恒等变换,即$\mathbf{x}\left(\mathbf{p}\right)=\mathbf{p},\forall\mathbf{p}\in\mathbb{R}^n$,则由点$\mathbf{p}_0$上的全微分引出的函数$d\mathbf{x}_{\mathbf{p}_0}:\mathbb{R}^n\rightarrow\mathbb{R}^n$满足$d\mathbf{x}_{\mathbf{p}_0}\left(\mathbf{u}\right)=\mathbf{M}_{\mathbf{p}_0}\mathbf{u}=\mathbf{I}_n\mathbf{u}=\mathbf{u}$,其中$\mathbf{M}_{\mathbf{p}_0}$是$\mathbf{x}$在$\mathbf{p}_0$处的导数,显然这个导数总为恒等线性变换$\mathbf{I}_n$,故$d\mathbf{x}_{\mathbf{p}_0}$也是$\mathbb{R}^n$上的恒等映射。我们把它的分量写出来:
\[
d\mathbf{x}_{\mathbf{p}_0}\left(\mathbf{u}\right)=\left(\begin{array}{c}
dx_{1,\mathbf{p}_0}\left(\mathbf{u}\right)\\
\vdots\\
dx_{n,\mathbf{p}_0}\left(\mathbf{u}\right)\end{array}\right)=\left(\begin{array}{c}
u_1\\\vdots\\u_n\end{array}\right)
\]
其中$dx_{i,\mathbf{p}_0}\left(\mathbf{u}\right)=u_i,i=1,\cdots,n$是取坐标函数,且它们都是当$\mathbf{u}\to\mathbf{0}$时的无穷小。因此$d\mathbf{x}_{\mathbf{p}_0}$是由这$n$个无穷小量组成的无穷小向量。现再考虑复合函数$\mathbf{f}\circ\mathbf{x}$在点$\mathbf{p}_0$处的全微分引出的函数,它满足\footnote{这里提前用到了复合函数求导法则。}
\[
d\left(\mathbf{f}\circ \mathbf{x}\right)_{\mathbf{p}_0}\left(\mathbf{u}\right)=\mathbf{L}_{\mathbf{x}\left(\mathbf{p}_0\right)}\mathbf{M}_{\mathbf{p}_0}\mathbf{u}=\mathbf{L}_{\mathbf{x}\left(\mathbf{p}_0\right)}d\mathbf{x}_{\mathbf{p}_0}\left(\mathbf{u}\right),\forall\mathbf{u}\in\mathbb{R}^n\]
由于线性变换是连续函数,故可证函数$d\left(\mathbf{f}\circ \mathbf{x}\right)_{\mathbf{p}_0}\left(\mathbf{u}\right)$与函数$\mathbf{x}_{\mathbf{p}_0}\left(\mathbf{u}\right)$当$\mathbf{u}\to\mathbf{0}$时都趋于零,且对任意$\mathbf{p}_0,\mathbf{u}\in\mathbb{R}^n$均成立,故
可略去$\mathbf{p}_0$和$\mathbf{u}$简记为
\[d\mathbf{f}\left(\mathbf{x}\right)=\mathbf{L}_\mathbf{x}d\mathbf{x}\]
其中
\[d\mathbf{x}=\left(\begin{array}{c}dx_1\\\vdots\\dx_n\end{array}\right)\]
该式就是函数的微分的定义式,由上述的讨论我们知道这个式子表达了一个无穷小向量与另一个无穷小向量之间的线性关系。

我们进一步把$\mathbf{L}_{\mathbf{x}}$记为$\frac{d\mathbf{f}}{d\mathbf{x}}$确实是滥用了符号,因为线性变换并非“两个向量的商”。但这种记法是很多资料都使用的惯例。与$\mathbf{L}_\mathbf{x}$相比更利于直接告诉我们这是一个函数的导数而无需另作文字说明,因此本讲义也经常使用这种记法,即
\[
d\mathbf{f}\left(\mathbf{x}\right)=\frac{d\mathbf{f}}{d\mathbf{x}}d\mathbf{x}\]

%================================================================
\subsection{向量函数的导函数、连续可微函数}
\begin{definition}[向量函数的导函数]
设函数$\mathbf{f}:\mathbb{R}^n\subseteq D\rightarrow\mathbb{R}^m$是开集$S\subseteq D$上的可微函数,则函数$\mathbf{f}$在$S$上的导函数(derivative function)是一个线性变换值函数$\mathbf{L}:S\rightarrow\mathcal{L}\left(\mathbb{R}^n,\mathbb{R}^m\right)$,
\[
\mathbf{L}\left(\mathbf{x}\right)\equiv\left.\frac{d\mathbf{f}\left(\mathbf{x}^\prime\right)}{d\mathbf{x}^\prime}\right|_{\mathbf{x}^\prime=\mathbf{x}},\forall\mathbf{x}\in S
\]
\end{definition}

注意,向量函数的导函数的函数值是线性变换。向量函数在不同点上的导数是不同的线性变换。在上述定义中的“$\mathbf{L}\left(\mathbf{x}\right)$”,不是指一个线性变换作用于一个向量,而是一个线性变换值函数及其自变量。此时$\mathbf{L}\left(\mathbf{x}\right)$可以作用于一个向量$\mathbf{u}$得到另一个向量$\mathbf{v}=\mathbf{L}\left(\mathbf{x}\right)\mathbf{u}$。到底$\mathbf{L}$后面的括号表示哪种意义,在记法上无法区分,但是大多数情况下可根据语境区分。本讲义在必要处会说明一种记法到底表示哪种意义。

由定理\ref{thm:II.12.4}知,导函数是单射。

接下来我们准备讨论导函数的连续性,这需要先引入“线性变换的范”的概念。

由定理\ref{thm:II.11.3},对任一线性变换$\mathbf{L}:\mathbb{R}^n\rightarrow\mathbb{R}^m$皆存在一个正实数$k>0$满足$\left\|\mathbf{Lx}\right\|\leq k\left\|\mathbf{x}\right\|,\forall\mathbf{x}\in\mathbb{R}^n$。于是我们定义——
\begin{definition}[线性变换的范]
设$\mathcal{V},\mathcal{W}$是数域$\mathbb{F}$上的赋范向量空间,线性变换$\mathbf{L}:\mathcal{V}\rightarrow\mathcal{W}$的范$\left\|\mathbf{L}\right\|$为满足$\left\|\mathbf{Lx}\right\|\leq k\left\|\mathbf{x}\right\|,\forall\mathbf{x}\in\mathbb{R}^n$的正实数$k$的最大下界(infimum),即
\[\left\|\mathbf{L}\right\|\equiv\mathrm{inf}\left\{k|k>0,\left\|\mathbf{Lx}\right\|\leq k\left\|\mathbf{x}\right\|\forall\mathbf{x}\right\}\]
\end{definition}

这一最大下界的存在性和唯一性是显然的。我们无需知道它具体数值。可验证,以上定义的范符合范的一般要求。又由于不同的范的定义是等价的(定理\ref{thm:VI.1.1}),故后续命题的证明过程每当需要线性变换的范的定义时都不妨通过上述这种范的定义来求证。由此定义我们可立即获得性质:$\left\|\mathbf{Lx}\right\|\leq\left\|\mathbf{L}\right\|\left\|\mathbf{x}\right\|$。

有了向量函数的导函数的定义以及线性变换的范的定义,我们可以很容易理解导函数的连续性是什么意思。按照函数连续性的定义,若函数$\mathbf{L}:S\rightarrow\mathcal{L}\left(\mathbb{R}^n,\mathbb{R}^m\right)$在$S$上连续,则对任意$\epsilon>0$总存在$\delta>0$使得只要$\left\|\mathbf{x}-\mathbf{x}_0\right\|<\delta,\mathbf{x},\mathbf{x}_0\in S$就有$\left\|\mathbf{L}\left(\mathbf{x}\right)-\mathbf{L}\left(\mathbf{x}_0\right)\right\|<\epsilon$。

下面介绍函数“连续可微”的概念,它大致上说的是:导函数连续,则函数连续可微。准确定义如下。

\begin{definition}[连续可微函数]
若函数$\mathbf{f}:\mathbb{R}^n\supseteq D\rightarrow\mathbb{R}^m$在开集$S\subseteq D$上有导函数$\mathbf{L}:S\rightarrow\mathcal{L}\left(\mathbb{R}^n,\mathbb{R}^m\right)$,且其为$S$上的连续函数,则称函数$\mathbf{f}\left(\mathbf{x}\right)$是$S$上的连续可微(continuously differentiable)函数。
\end{definition}

\begin{corollary}
函数$\mathbf{f}:\mathbb{R}^n\supseteq D\rightarrow\mathbb{R}^m$在开集$D$上连续可微当且仅当函数$\mathbf{f}$在$D$上的偏导数
\[
\left.\frac{\partial f_i\left(\mathbf{x}\right)}{\partial x_j}\right|_{\mathbf{x}\in D},i=1,\cdots,m,j=1,\cdots,n
\]
都存在且连续。
\end{corollary}

结合这一推论和定理\ref{thm:II.11.5}可知函数连续可微是函数可微的充份非必要条件。

%======================================================
\subsection{对向量的导数与方向导数}
刚才在介绍雅可比矩阵的时候,我们利用全微分的性质考虑过以下极限:
\[
\lim_{t\to 0}\frac{\mathbf{f}\left(\mathbf{x}_i\right)-\mathbf{f}\left(\mathbf{x}_0\right)}{t},i=1,\cdots,n\]
并知道它就是$\mathbf{f}\left(\mathbf{x}\right)$在$\mathbf{x}_0=\left(x_{01},\cdots,x_{0n}\right)^\intercal$处对$x_{0i}$的偏导数,因为上式$\Leftrightarrow$
\[
\lim_{t\rightarrow 0}\frac{\mathbf{f}\left(\mathbf{x}_0+t\mathbf{\hat{e}}_i\right)-\mathbf{f}\left(\mathbf{x}_0\right)}{t},i=1,\cdots,n
\]
现在我们把$\mathbf{\hat{e}}_j$改为任意向量$\mathbf{y}$,引入方向导数。

\begin{definition}[对向量的导数、方向导数]
函数$\mathbf{f}:\mathbb{R}^n\rightarrow\mathbb{R}^m$对向量$\mathbf{y}\in\mathbb{R}^n$的导数是
\[\frac{\partial\mathbf{f}\left(\mathbf{x}\right)}{\partial\mathbf{y}}=\lim_{t\to 0}\frac{\mathbf{f}\left(\mathbf{x}+t\mathbf{y}\right)-\mathbf{f}\left(\mathbf{x}\right)}{t}\]
其中作为导函数的$\frac{\partial \mathbf{f}\left(\mathbf{x}\right)}{\partial\mathbf{y}}$的定义域是原函数$\mathbf{f}$的定义域的使该导数存在的子集。特别地,函数对单位向量$\mathbf{u},\left\|\mathbf{u}\right\|=1$的导数称为方向导数。
\end{definition}

以下定理使得函数对任意向量的导数可用该函数的全导数来计算。

\begin{theorem}\label{thm:II.11.6}
设函数$\mathbf{f}:\mathbb{R}^n\rightarrow\mathbb{R}^m$在$\mathbf{x}\in\mathbb{R}^n$处可微,则
\[\frac{\partial\mathbf{f}\left(\mathbf{x}\right)}{\partial\mathbf{y}}=\frac{d\mathbf{f}\left(x\right)}{d\mathbf{x}}\mathbf{y},\forall\mathbf{y}\in\mathbb{R}^n\]
\end{theorem}
\begin{proof}
对$\mathbf{y}=\mathbf{0}$显然成立。若$\mathbf{y}\neq\mathbf{0}$,由全微分的近似意义,由自变量的增量$t\mathbf{y}$造成的函数增量$\mathbf{f}\left(\mathbf{x}+t\mathbf{y}\right)-\mathbf{f}\left(\mathbf{x}\right)$满足
\begin{align*}
\lim_{t\to 0}\frac{\mathbf{f}\left(\mathbf{x}+t\mathbf{y}\right)-\mathbf{f}\left(\mathbf{x}\right)-\frac{d\mathbf{f}\left(\mathbf{x}\right)}{d\mathbf{x}}\left(t\mathbf{y}\right)}{\left\|t\mathbf{y}\right\|}&=\mathbf{0}\\
\Leftrightarrow\lim_{t\to 0}\frac{1}{\left\|\mathbf{y}\right\|}\left\|\frac{\mathbf{f}\left(\mathbf{x}+t\mathbf{y}\right)-\mathbf{f}\left(x\right)}{t}-\frac{d\mathbf{f}\left(\mathbf{x}\right)}{d\mathbf{x}}\mathbf{y}\right\|&=\mathbf{0}\\
\Leftrightarrow\lim_{t\to 0}\frac{\mathbf{f}\left(\mathbf{x}+t\mathbf{y}\right)-\mathbf{f}\left(\mathbf{x}\right)}{t}&=\frac{d\mathbf{f}\left(\mathbf{x}\right)}{d\mathbf{x}}\mathbf{y}
\end{align*}
\end{proof}

我们比较一个函数$\mathbf{f}\left(\mathbf{x}\right)$在$\mathbf{x}_0$处的全微分及其在$\mathbf{x}_0$处对某向量$\mathbf{y}$的导数可以发现后者就是令前者中的$\mathbf{x}-\mathbf{x}_0=\mathbf{y}$。在全微分中,我们强调的是对任意$\mathbf{x}$(即点$\mathbf{x}_0$邻近的每处),而在对$\mathbf{y}$的导数中我们强调的是$\mathbf{x}_0$朝某个选定的向量$\mathbf{y}$的某处($\mathbf{x}=\mathbf{x}_0+\mathbf{y}$)。留意到,函数在某点处对标准基向量的导数就是函数对该点相应分量的偏导数。

方向导数的几何意义是函数在某点处朝相应方向的变化率。定理\ref{thm:II.16.6}表明,拿一个函数在某点处的全导数作用于一个单位向量,就可以得该函数在该点处朝该方向的变化率。这也是函数的全导数的重要几何意义。
%===============================================
\subsection{复合函数求导的链式法则、反函数定理、隐函数定理}
\begin{theorem}[复合函数求导的链式法则]\label{thm:II.12.7}
如果函数$\mathbf{f}:\mathbb{R}^n\supseteq D\rightarrow\mathbb{R}^m$在$\mathbf{x}_0\in D$处可微分;函数$\mathbf{g}:\mathbb{R}^n\supset E\rightarrow\mathbb{R}^p$在$\mathbf{f}\left(\mathbf{x}_0\right)\in E\cap D$处可微分,则复合函数$\mathbf{g}\circ\mathbf{f}$在$\mathbf{x}_0$处可微分,且其全导数
\[
\left.\frac{d\mathbf{g}\circ\mathbf{f}\left(\mathbf{x}\right)}{d\mathbf{x}}\right|_{\mathbf{x}=\mathbf{x}_0}=\left.\frac{d\mathbf{g}\left(\mathbf{y}\right)}{d\mathbf{y}}\right|_{\mathbf{y}=\mathbf{f}\left(\mathbf{x}_0\right)}\left.\frac{d\mathbf{f}\left(\mathbf{x}\right)}{d\mathbf{x}}\right|_{\mathbf{x}=\mathbf{x}_0}
\]
\end{theorem}
\begin{proof}
见附录。
\end{proof}

\begin{theorem}[反函数定理]\label{thm:II.12.8}
设函数$\mathbf{f}:\mathbb{R}^n\rightarrow\mathbb{R}^n$是连续可微函数,且在$\mathbf{x}_0$处其导数$\mathbf{L}\left(\mathbf{x}_0\right)$可逆。则在$\mathbf{x}_0$的某邻域$N$上,函数$\mathbf{f}$有连续可微的逆函数$\mathbf{f}^{-1}$;$N$的像集$\mathbf{f}\left(\mathbf{N}\right)$是开集;且$\mathbf{f}^{-1}$在$\mathbf{f}\left(\mathbf{x}_0\right)$处的导数是$\mathbf{f}$在$\mathbf{x}_0$的导数的逆变换,即
\[\left.\frac{d\mathbf{f}^{-1}\left(\mathbf{y}\right)}{d\mathbf{y}}\right|_{\mathbf{y}=\mathbf{f}\left(\mathbf{x}_0\right)}=\mathbf{L}^{-1}\left(\mathbf{x}_0\right)\]
\end{theorem}
\begin{proof}
见附录
\end{proof}

如果$\mathbf{f}:\mathbb{R}^n\rightarrow\mathbb{R}^m$由函数$\mathbf{F}:\mathbb{R}^{n+m}\rightarrow\mathbb{R}^m$隐含定义,我们常常把$\mathbf{F}$写成这样一种映射:$\mathbf{F}:\mathbb{R}^n\times\mathbb{R}^m\rightarrow\mathbb{R}^m$,$\mathbf{F}\left(\mathbf{x},\mathbf{f}\left(\mathbf{x}\right)\right)=\mathbf{0}\forall\mathbf{x}\in\mathbb{R}^n$。此时,我们记
\[\left.\frac{\partial \mathbf{F}\left(\mathbf{x},\mathbf{y}\right)}{\partial \mathbf{x}}\right|_{\mathbf{x}=\mathbf{x}_0,\mathbf{y}=\mathbf{y}_0}=\left.\frac{d\mathbf{F}\left(\mathbf{x},\mathbf{y}_0\right)}{d\mathbf{x}}\right|_{\mathbf{x}=\mathbf{x}_0}\]
引入这个记法有两个用处,一是为了以下例子,这个例子是后文引入物质导数的一个基础;二是为了引入隐函数定理。

\begin{example}\label{exp:II.12.1}
设函数$\mathbf{f}:\mathbb{R}^{n+m}\rightarrow\mathbb{R}^p$的定义域为$D=\mathrm{dom}\mathbf{f}$,若函数$\mathbf{g}:\mathbb{R}^m\rightarrow\mathbb{R}^n$满足$\mathrm{dom}\mathbf{g}=D$,则总可以构建函数$\mathbf{h}:\mathbb{R}^m\rightarrow\mathbb{R}^{m+n}$使得$\mathrm{dom}\mathbf{h}=D$且
\begin{align*}
    \mathbf{h}\left(\mathbf{x}\right)&=\left(\mathbf{g}\left(\mathbf{x}\right),\mathbf{x}\right)\forall\mathbf{x}\in D\\
    h_i\left(\mathbf{x}\right)&=\left\{\begin{array}{ll}
    g_i\left(\mathbf{x}\right),&i=1,\cdots,n\\
    x_{i-n},&i=n+1,\cdots,n+m
    \end{array}\right.
\end{align*}
这时,$\mathbf{f}\left(\mathbf{g}\left(\mathbf{x}\right),\mathbf{x}\right)=\mathbf{f}\left(\mathbf{h}\left(\mathbf{x}\right)\right)$。若$\mathbf{g}$在某处可微分则$\mathbf{h}$在该处也可微分。由链式法则定理,在该处有
\begin{align*}
    \frac{d\mathbf{f}\left(\mathbf{g}\left(\mathbf{x}\right),\mathbf{x}\right)}{d\mathbf{x}}&=\frac{d\mathbf{h}\left(\mathbf{x}\right)}{d\mathbf{x}}\\
    &=\left.\frac{d\mathbf{f}\left(\mathbf{y}\right)}{d\mathbf{y}}\right|_{\mathbf{y}=\mathbf{h}\left(\mathbf{x}\right)}\frac{d\mathbf{h}\left(\mathbf{x}\right)}{d\mathbf{x}}\\
    &=\left(\begin{array}{cccccc}
    \frac{\partial f_1}{\partial h1}&\cdots&\frac{\partial f_1}{\partial h_n}&\frac{\partial f_1}{\partial h_{n+1}}&\cdots&\frac{\partial f_1}{\partial h_{n+m}}\\
    \vdots&\ddots&\vdots&\vdots&\ddots&\vdots\\
    \frac{\partial f_p}{\partial h_1}&\cdots&\frac{\partial f_p}{\partial h_n}&\frac{\partial f_p}{\partial h_{n+1}}&\cdots&\frac{\partial f_p}{\partial h_{n+m}}
    \end{array}\right)\left(\begin{array}{ccc}
    \frac{\partial h_1}{\partial y_1}&\cdots&\frac{\partial h_1}{\partial y_m}\\
    \vdots&\ddots&\vdots\\
    \frac{\partial h_n}{\partial y_1}&\cdots&\frac{\partial h_1}{\partial y_m}\\
    \frac{\partial h_{n+1}}{\partial y_1}&\cdots&\frac{\partial h_{n+1}}{\partial y_m}\\
    \vdots&\ddots&\vdots\\
    \frac{\partial h_{n+m}}{\partial y_1}&\cdots&\frac{\partial h_{n+m}}{\partial y_m}
    \end{array}\right)
    &=C
\end{align*}
其中矩阵$C$的分量
\[
    C_{ij}=\sum_{k=1}^{n+m}\frac{\partial f_i}{\partial h_k}\frac{\partial h_k}{\partial x_j},i=1,\cdots,p,j=1,\cdots,m
\]
注意到
\begin{align*}
    \frac{\partial f_i}{\partial h_k}\frac{\partial h_k}{\partial x_j}=\left\{\begin{array}{ll}
    \frac{\partial f_i}{\partial g_k}\frac{\partial g_k}{x_j},&k=1,\cdots,n\\
    \frac{\partial f_i}{\partial x_k-n}\delta_{k-n,j},&k=n+1,\cdots,n+m
    \end{array}\right.,i=1,\cdots,p,j=1,\cdots,m
\end{align*}
故
\begin{align*}
C_{ij}&=\sum_{k=1}^{n}\frac{\partial f_i}{\partial g_k}\frac{\partial g_k}{\partial x_j}+\sum_{j=1}^m\frac{\partial f_i}{\partial x_j}\\
\Leftrightarrow\frac{d\mathbf{f}\left(\mathbf{g}\left(\mathbf{x}\right),\mathbf{x}\right)}{d\mathbf{x}}&=\left.\frac{\partial \mathbf{f}\left(\mathbf{y},\mathbf{x}\right)}{\partial \mathbf{y}}\right|_{\mathbf{y}=\mathbf{g}\left(\mathbf{x}\right)}\frac{\partial\mathbf{g}\left(\mathbf{x}\right)}{\partial \mathbf{x}}+\frac{\partial \mathbf{f}\left(\mathbf{y},\mathbf{x}\right)}{\partial\mathbf{x}}
\end{align*}
\end{example}

\begin{theorem}[隐函数定理]\label{thm:II.12.9}
设$\mathbf{F}:\mathbb{R}^{n+m}\rightarrow\mathbb{R}^m$是连续可导函数,且对$\mathbf{x}_0\in\mathbf{R}^n$和$\mathbf{y}_0\in\mathbf{R}^m$有
\begin{itemize}
    \item $\mathbf{F}\left(\mathbf{x}_0,\mathbf{y}_0\right)=\mathbf{0}$;
    \item 导数$\left.\frac{\partial \mathbf{F}\left(\mathbf{x},\mathbf{y}\right)}{\partial \mathbf{y}}\right|_{\mathbf{x}=\mathbf{x}_0,\mathbf{y}=\mathbf{y}_0}$可逆;
\end{itemize}
则$\mathbf{x}_0$的某邻域$N$存在由$\mathbf{F}$隐函定义的连续可微函数$\mathbf{f}:\mathbb{R}^n\rightarrow\mathbb{R}^m$满足$\mathbf{F}\left(\mathbf{x},\mathbf{f}\left(\mathbf{x}\right)\right))=\mathbf{0}\forall\mathbf{x}\in N$,且在$N$上有
\[\left.\frac{d\mathbf{f}\left(\mathbf{x}\right)}{d\mathbf{x}}\right|_{\mathbf{x}=\mathbf{x}}=-\left[\left.\frac{\partial\mathbf{F}\left(\mathbf{x},\mathbf{y}\right)}{\partial\mathbf{y}}\right|_{\mathbf{x}=\mathbf{x},\mathbf{y}=\mathbf{f}\left(\mathbf{x}\right)}\right]^{-1}\left.\frac{\partial\mathbf{F}\left(\mathbf{x},\mathbf{y}\right)}{\partial\mathbf{x}}\right|_{\mathbf{x}=\mathbf{x},\mathbf{y}=\mathbf{f}\left(\mathbf{x}\right)}\]
上式中的“$-1$”是指线性变换的逆。
\end{theorem}
\begin{proof}
待补充\cite[p.~593]{Williamson1972}。
\end{proof}

\begin{example}\label{exp:II.12.2}
设函数$\mathbf{F}:\mathbb{R}^4\rightarrow\mathbb{R}^2$,$\mathbf{F}\left(u,v,x,y\right)=\left(F_1,F_2\right)$,$\mathbf{F}=\mathbf{0}$隐含定义了$\left(x,y\right)=\mathbf{f}\left(u,v\right),\mathbf{f}:\mathbf{R}^2\rightarrow\mathbf{R}^2$,则
\begin{align*}
    &\frac{\partial \mathbf{F}}{\partial u}=\mathbf{0},\frac{\partial \mathbf{F}}{\partial v}=\mathbf{0}\\
    \Leftrightarrow&\left\{\begin{array}{l}
    \frac{\partial F_1}{\partial u}+\frac{\partial F_1}{\partial x}\frac{\partial x}{\partial u}+\frac{\partial F_1}{\partial y}\frac{\partial y}{\partial u}=0\\
    \frac{\partial F_2}{\partial u}+\frac{\partial F_2}{\partial x}\frac{\partial x}{\partial u}+\frac{\partial F_2}{\partial y}\frac{\partial y}{\partial u}=0
    \end{array}\right.,\left\{\begin{array}{l}
    \frac{\partial F_1}{\partial v}+\frac{\partial F_1}{\partial x}\frac{\partial x}{\partial v}+\frac{\partial F_1}{\partial y}\frac{\partial y}{\partial v}=0\\
    \frac{\partial F_2}{\partial v}+\frac{\partial F_2}{\partial x}\frac{\partial x}{\partial v}+\frac{\partial F_2}{\partial y}\frac{\partial y}{\partial v}=0
    \end{array}\right.\\
    \Leftrightarrow&\left(\begin{array}{cc}
    \frac{\partial F_1}{\partial u}&\frac{\partial F_1}{\partial v}\\\frac{\partial F_2}{\partial u}&\frac{\partial F_2}{\partial v}\end{array}\right)_+\left(\begin{array}{cc}
    \frac{\partial F_1}{\partial x}&\frac{\partial F_1}{\partial y}\\\frac{\partial F_2}{\partial x}&\frac{\partial F_2}{\partial y}\end{array}\right)\left(\begin{array}{cc}
    \frac{\partial x}{\partial u}&\frac{\partial x}{\partial v}\\\frac{\partial y}{\partial u}&\frac{\partial y}{\partial v}\end{array}\right)=0\\
    \Leftrightarrow&\left(\begin{array}{cc}
    \frac{\partial x}{\partial u}&\frac{\partial x}{\partial v}\\\frac{\partial y}{\partial u}&\frac{\partial y}{\partial v}\end{array}\right)=-\left(\begin{array}{cc}
    \frac{\partial F_1}{\partial x}&\frac{\partial F_1}{\partial y}\\\frac{\partial F_2}{\partial x}&\frac{\partial F_2}{\partial y}\end{array}\right)^{-1}\left(\begin{array}{cc}
    \frac{\partial F_1}{\partial u}&\frac{\partial F_1}{\partial v}\\\frac{\partial F_2}{\partial u}&\frac{\partial F_2}{\partial v}\end{array}\right)
\end{align*}
\end{example}


\end{document}