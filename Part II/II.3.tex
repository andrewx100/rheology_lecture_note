\documentclass[main.tex]{subfiles}
% 内积空间
\begin{document}
在本节中,我们在向量空间的基础上再增加一些性质和运算法则,使得“正交基”、“单位向量”的概念得以引入,同时介绍其他重要概念。

\begin{definition}[内积]
数域$\mathbb{F}$上的向量空间$\mathcal{V}$中两向量$\mathbf{a},\mathbf{b}\in\mathcal{V}$的内积(inner product)记为$\left(\mathbf{a}|\mathbf{b}\right),\left(\cdot|\cdot\right):\mathcal{V}\times\mathcal{V}\rightarrow\mathbb{F}$,满足\footnote{上划线表示复数共轭}:$\forall\mathbf{a},\mathbf{b},\mathbf{c}\in\mathcal{V},\alpha\in\mathbb{F}$
\begin{itemize}
    \item $\left(\mathbf{a}|\mathbf{b}\right)=\overline{\left(\mathbf{b}|\mathbf{a}\right)}$
    \item $\left(\alpha\mathbf{a}|\mathbf{b}\right)=\alpha\left(\mathbf{a}|\mathbf{b}\right)$
    \item $\left(\mathbf{a}+\mathbf{b}|\mathbf{c}\right)=\left(\mathbf{a}|\mathbf{b}\right)+\left(\mathbf{a}|\mathbf{c}\right)$
    \item $\left(\mathbf{a}|\mathbf{a}\right)\in\mathbb{R}$且$\left(\mathbf{a}|\mathbf{a}\right)\geq0$,当且仅当$\mathbf{a}=\mathbf{0}$时取等号。
\end{itemize}
带有内积运算规定的向量空间叫做内积空间(inner product space)。
\end{definition}

由内积运算可推出以下性质:$\forall\mathbf{a},\mathbf{b},\mathbf{c}\in\mathcal{V},\alpha\in\mathbb{F}$
\[\left(\mathbf{a}|\alpha\mathbf{b}+\mathbf{c}\right)=\overline{\alpha}\left(\mathbf{a}|\mathbf{b}\right)+\left(\mathbf{a}|\mathbf{c}\right)\]
即从内积的后一个向量提出标量到内积之外时,这一标量要取复数共轭。

内积的定义中设置复数共轭是必要的。否则将面临如下的矛盾:由$\left(\mathbf{a}|\mathbf{a}\right)>0\forall\mathbf{a}\neq\mathbf{0}$,竟然有$\left(i\mathbf{a}|i\mathbf{a}\right)=-1\left(\mathbf{a}|\mathbf{a}\right)>0$。但是,到底是规定从内积的前一个向量提出的标量要取复数共轭(很多物理书习惯),还是从内积的后一个向量提出的标量要取复数共轭(本讲义的定义方式,也是很多数学书的习惯)。这个惯例的不同,在数学上有时会造成重大的后果,但在流变学中并不重要。我们将会在线性泛函与对偶空间的时候再讲这件事,不感兴趣的同学可以忽略。

\begin{example}
在$\mathbb{F}^n$上可定义这样的内积:对于$\mathbf{a}=\left(\alpha_1,\cdots,\alpha_n\right)^\intercal,\mathbf{b}=\left(\beta_1,\cdots,\beta_n\right)^\intercal\in\mathbb{F}$,$\left(\mathbf{a}|\mathbf{b}\right)\equiv\sum_j\alpha_j\overline{\beta_j}$,称为$\mathbb{F}^n$上的标准内积(standard inner product)。$\mathbb{R}^n$上的标准内积又可记为点乘(dot product)$\mathbf{a}\cdot\mathbf{b}$。
\end{example}

由内积的一般定义可证明柯西--施瓦茨不等式(Cauchy--Schwarz inequality),故该不等式对任一满足内积定义的一般要求的具体定义都成立。

\begin{theorem}\label{thm:II.3.1}
设$\mathcal{V}$是数域$\mathbb{F}$上的一个内积空间,则有$\forall\mathbf{a},\mathbf{b}\in\mathcal{V}$,
\begin{enumerate}
    \item 柯西--施瓦茨不等式$\left|\left(\mathbf{a}|\mathbf{b}\right)\right|\leq\left(\mathbf{a}|\mathbf{a}\right)\left(\mathbf{b}|\mathbf{b}\right)$
    \item $\left(\mathbf{a}|\mathbf{b}\right)=\mathrm{Re}\left(\mathbf{a}|\mathbf{b}\right)+\mathrm{Re}\left(\mathbf{a}|i\mathbf{b}\right)$
\end{enumerate}
\end{theorem}
\begin{proof}
当$\mathbf{a}=\mathbf{0}$时,不等式取等号成立。当$\mathbf{a}\neq\mathbf{0}$时,令
\[
\mathbf{c}=\mathbf{b}-\frac{\left(\mathbf{b}|\mathbf{a}\right)}{\left(\mathbf{a}|\mathbf{a}\right)}\mathbf{a}
\]
则可验证$\left(\mathbf{c}|\mathbf{a}\right)=0$,且
\begin{align*}
0\leq\left(\mathbf{c}|\mathbf{c}\right)&=\left(\mathbf{b}-\frac{\left(\mathbf{b}|\mathbf{a}\right)}{\left(\mathbf{a}|\mathbf{a}\right)}\mathbf{a}\right|\left.\mathbf{b}-\frac{\left(\mathbf{b}|\mathbf{a}\right)}{\left(\mathbf{a}|\mathbf{a}\right)}\mathbf{a}\right)\\
&=\left(\mathbf{b}|\mathbf{b}\right)-\frac{\left|\left(\mathbf{b}|\mathbf{a}\right)\right|^2}{\left(\mathbf{a}|\mathbf{a}\right)}\\
\Leftrightarrow \left|\left(\mathbf{a}|\mathbf{b}\right)\right|&\leq\left(\mathbf{a}|\mathbf{a}\right)\left(\mathbf{b}|\mathbf{b}\right)
\end{align*}

由$\left(\mathbf{a}|\mathbf{b}\right)=\mathrm{Re}\left(\mathbf{a}|\mathbf{b}\right)+i\mathrm{Im}\left(\mathbf{a}|\mathbf{b}\right)$和$\mathrm{Im}\left(\alpha\right)=\mathrm{Re}\left(-i\alpha\right)\forall\alpha\in\mathbb{F}$,有$\mathrm{Im}\left(\mathbf{a}|\mathbf{b}\right)=\mathrm{Re}\left(-i\left(\mathbf{a}|\mathbf{b}\right)\right)=\mathrm{Re}\left(\mathbf{a}|i\mathbf{b}\right)$,故$\left(\mathbf{a}|\mathbf{b}\right)=\mathrm{Re}\left(\mathbf{a}|\mathbf{b}\right)+i\mathrm{Re}\left(\mathbf{a}|i\mathbf{b}\right)$。
\end{proof}

除了内积空间,我们还可以为一个向量空间引入范(norm)的规定,得到赋范向量空间(normed vector space)。

\begin{definition}[向量的范]\label{def:II.3.2}
设$\mathcal{V}$是数域$\mathbb{F}$向量空间,则有$\forall\mathbf{x}\in\mathcal{V},\alpha\in\mathbb{F}$,
\begin{enumerate}
    \item 非负性:$\left\|\mathbf{x}\right\|\geq 0$
    \item 调和性:$\left\|\alpha\mathbf{x}\right\|=\left|\alpha\right|\left\|\mathbf{x}\right\|$
    \item 三角不等式:$\left\|\mathbf{x}+\mathbf{y}\right\|\leq\left\|\mathbf{x}\right\|+\left\|\mathbf{y}\right\|$
\end{enumerate}
带有范的定义的向量空间叫赋范向量空间。
\end{definition}

我们注意到,赋范向量空间也有一个总成立的不等式——三角不等式。与内积空间的柯西--施瓦茨不等式不同,赋范向量空间的三角不等式是在范的定义中直接规定的,而无法法作为定理由之前两个规定证明出来。这说明我们主观上就希望向量的范满足这样的性质。这是因为,“范”是我们给于向量以“长度”的概念,并希望它能于欧几里德几何公设规定的性质相兼容。

一个赋范向量空间$\mathcal{V}$中,范为1的向量称为单位向量(unit vector),在本讲义中单位向量会加一个小帽子来特别表示:$\left\|\hat{\mathbf{a}}\right\|=1,\hat{\mathbf{a}}\in\mathcal{V}$。赋范向量空间$\mathcal{V}$的任一向量$\mathcal{x}$均可通过$\hat{\mathbf{x}}=\mathbf{x}/\left\|\mathbf{x}\right\|$归一化为一个单位向量(由范的定义易验)。

不管是内积的定义、还是范的定义,都没有具体规定计算方法。只要满足相应定义中的一般要求的任一种定义,都可作为内积或范。向量的范的其中一种常用的定义是:设$\mathcal{V}$是内积空间,向量$\mathbf{a}\in\mathcal{V}$的范$\left\|\mathbf{a}\right\|\equiv\left(\mathbf{a}|\mathbf{a}\right)^{\frac{1}{2}}$。我们把这一定义称为欧几里德范。其他范的定义则为非欧几里德范,例如在$\mathbb{R}^n$上,还可以有如下范的定义。对任一$\mathbf{x}=\left(x_1,\cdots,x_n\right)$,$\left\|\mathbf{x}\right\|=\mathrm{max}\left\{\left|x_1\right|,\cdots,\left|x_n\right|\right\}$。一般地,由于给一个向量空间引入内积定义的方式本身就可以有多种,依赖内积的范的定义也会有多种。对于有些向量空间,范的定义可以不依赖内积。

以下定理说明,在一定条件下,我们能够用范的一般定义构造一个内积,使任何一个尚未定义内积的赋范空间成为一个内积空间。

\begin{theorem}\label{thm:II.3.2}
一个赋范向量空间是内积空间当且仅当该空间的范满足极化恒等式(polarization identity),即
\[\left\|\mathbf{a}+\mathbf{b}\right\|^2+\left\|\mathbf{a}-\mathbf{b}\right\|^2=2\left\|\mathbf{a}\right\|^2+2\left\|\mathbf{b}\right\|^2\]
\end{theorem}
\begin{proof}
设$\mathcal{V}$是数域$\mathbb{F}$上的一个赋范向量空间,若定义内积:
\begin{align*}
\left(\mathbf{a}|\mathbf{b}\right)&=\frac{1}{4}\left\|\mathbf{a}+\mathbf{b}\right\|^2-\frac{1}{4}\left\|\mathbf{a}-\mathbf{b}\right\|^2+\frac{i}{4}\left\|\mathbf{a}+i\mathbf{b}\right\|^2-\frac{i}{4}\left\|\mathbf{a}-i\mathbf{b}\right\|^2\\
&=\frac{1}{4}\sum_{n=1}^4i^n\left\|\mathbf{a}+i^n\mathbf{b}\right\|^2,\forall\mathbf{a},\mathbf{b}\in\mathcal{V}
\end{align*}
可验证上式满足内积定义,且$\left\|\mathbf{a}\right\|^2=\left(\mathbf{a}|\mathbf{a}\right)^{\frac{1}{2}}$。
\end{proof}

上面的等式在几何上等价于平行四边形法则(parallelogram law)。特别地,对于$\mathbf{a},\mathbf{b}\in\mathbb{R}^n$则有$\mathbf{a}\cdot\mathbf{b}=\frac{1}{4}\left\|\mathbf{a}+\mathbf{b}\right\|^2-\frac{1}{4}\left\|\mathbf{a}-\mathbf{b}\right\|^2$。

柯西--施瓦茨不等式、三角不等式和极化恒等式在很多定理的证明中经常用到,但它们的含义及适用范围需要区分清楚。由定理\ref{thm:II.3.1},柯西--施瓦茨不等式是对任一内积空间均成立的。由定义\ref{def:II.3.2},三角不等式是对任一赋范向量空间均成立的。而由定理\ref{thm:II.3.2}可知,满足极化恒等式的赋范空间必然也是一个内积空间从而把两种空间统一起来。特别地,欧几里德范总满足柯西--施瓦茨不等式、三角不等式和极化恒等式。

下面我们由内积空间的性质引入正交(orthogonal)及相关的概念。

\begin{definition}[正交]\label{def:II.3.3}
内积空间$\mathcal{V}$中的向量$\mathbf{a},\mathbf{b}\in\mathcal{V}$,若$\left(\mathbf{a}|\mathbf{b}\right)=0$,则称$\mathbf{a}$与$\mathbf{b}$是正交的。若$S$是$\mathcal{V}$的一个子集,且$S$中的向量两两正交,则称$S$是$\mathcal{V}$的一个正交集(orthogonal set)。若$\mathcal{V}$还是一个赋范向量空间,且$\mathcal{V}$的一个正交集$S$中的向量均满足$\left\|\hat{\mathbf{e}}\right\|=1\forall\hat{\mathbf{e}}\in S$则称$S$是规范正交集(orthonormal set)。
\end{definition}

\begin{example}\label{exp:II.3.2}\cite[“例题2.2”,p.~174]{周胜林2012线性代数}
数域$\mathbb{F}$上的$n\times n$矩阵的空间$\mathbb{F}^{n\times n}$中,记$E^{pq}$为仅第$p$行、第$q$列为1,其余为0的$n\times n$矩阵。则由$E^{pq},p=1,\cdots,n,q=1,\cdots,n$组成的集合是规范正交集。其中$\mathbb{F}^{n\times n}$上的内积定义是$\left(A|B\right)\equiv\sum_{j,k}A_{jk}\overline{B}_{jk},\forall A, B\in\mathbb{F}^{n\times n}$,$\overline{B}$表示矩阵$B$的每个分量均取复数共轭后的矩阵。
\end{example}

以下定理证明任意一组两两正交的向量线性无关,但需注意的是线性无关向量组却未必两两正交。

\begin{theorem}\label{thm:II.3.3}
正交集中的所有非零向量线性无关。
\end{theorem}
\begin{proof}
设$\mathcal{V}$是内积空间,$S$是$\mathcal{V}$的一个正交集,$\mathbf{a}_1,\cdots,\mathbf{a}_m\in S$。令$\mathbf{b}=\beta_1\mathbf{a}_1+\cdots+\beta_m\mathbf{a}_m$,则
\begin{align*}
\left(\mathbf{b}|\mathbf{a}_k\right)&=\left(\left.\sum_j\beta_j\mathbf{a}_j\right|\left.\mathbf{a}_k\right.\right)\\
&=\sum_j\beta_j\left(\mathbf{a}_j|\mathbf{a}_k\right)\\
&=\beta_j\left(\mathbf{a}_k|\mathbf{a}_k\right),k=1,\cdots,m\\
\because \left(\mathbf{a}_k|\mathbf{a}_k\right)&\neq 0\\
\therefore \beta_k&=\frac{\left(\mathbf{b}|\mathbf{a}\right)}{\left\|\mathbf{a}\right\|^2},k=1,\cdots,m
\end{align*}
考察上式可验证$\mathbf{b}=\mathbf{0}\Leftrightarrow\beta_1=\cdots=\beta_m=0$。
\end{proof}

由定理\ref{thm:II.3.3}以及子空间、线性生成空间的定义(\ref{def:II.2.2}、\ref{def:II.2.4}),内积空间$\mathcal{V}$的正交集$S$总能线性生成$\mathcal{V}$的一个子空间。若内积空间$\mathcal{V}$的一组基$\mathcal{B}$是正交集,则称$\mathcal{B}$为$\mathcal{V}$的正交基(orthogonal basis)。如果$\mathcal{V}$是赋范内积空间,其一组基$\mathcal{B}$是规范正交集,则称$\mathcal{B}$是$\mathcal{V}$的一个规范正交基(orthonormal basis)。

定理\ref{thm:II.3.3}的证明也给出了格拉姆--施密特正交化过程(Gram--Schmidt process),作为定理如下。

\begin{theorem}
设$\mathcal{V}$是内积空间,$\mathbf{b}_1,\cdots,\mathbf{b}_n\in\mathcal{V}$是一组线性无关向量。那么可以由它们构建一组两两正交的向量$\mathbf{a}_1,\cdots,\mathbf{a}_n\in\mathcal{V}$使得对于每一$k=1,\cdots,n$,向量组$\left\{\mathbf{a}_1,\cdots,\mathbf{a}_k\right\}$都是由$\left\{\mathbf{b}_1,\cdots,\mathbf{b}_k\right\}$线性生成的空间的一组基。
\end{theorem}
\begin{proof}
采用数学归纳法。作为$k=1$的情况,令$\mathbf{a}_1=\mathbf{b}_1$,则命题显然成立。假设当$k=m$时命题成立,即$\left\{\mathbf{a}_1,\cdots,\mathbf{a}_m\right\},m<n$是已经构建好的满足命题要求的正交向量,则对每一$k=1,\cdots,m$,$\left\{\mathbf{a}_1,\cdots,\mathbf{a}_k\right\}$是由$\left\{\mathbf{b}_1,\cdots,\mathbf{b}_k\right\}$线性生成的子空间的正交基。令
\[
\mathbf{a}_{m+1}=\mathbf{b}_{m+1}-\sum_{k=1}^m\frac{\left(\left.\mathbf{b}_{m+1}\right|\left.\mathbf{a}_k\right.\right)}{\left(\mathbf{a}_k|\mathbf{a}_k\right)}\mathbf{a}_k
\]
则有$\mathbf{a}_{m+1}\neq\mathbf{0}$,否则$\mathbf{b}_1,\cdots,\mathbf{b}_m,\mathbf{b}_{m+1}$线性相关,因$\mathbf{b}_{m+1}$可由$\mathbf{b}_1,\cdots,\mathbf{b}_m$线性表出。由上式还可知,对每一$j=1,\cdots,m$均有
\begin{align*}
\left(\mathbf{a}_{m+1}|\mathbf{a}_j\right)&=\left(\mathbf{b}_{m+1}|\mathbf{a}_j\right)-\sum_{k=1}^m\frac{\left(\mathbf{b}_{m+1}|\mathbf{a}_k\right)}{\left(\mathbf{a}_k|\mathbf{a}_k\right)}\left(\mathbf{a}_k|\mathbf{a}_j\right)\\
&=\left(\mathbf{b}_{m+1}|\mathbf{a}_j\right)-\left(\mathbf{b}_{m+1}|\mathbf{a}_j\right)\\
&=0
\end{align*}
所以$\left\{\mathbf{a}_1,\cdots,\mathbf{a}_{m+1}\right\}$是一个非零正交集。由定理\ref{thm:II.3.3},它们线性无关。故$\left\{\mathbf{a}_1,\cdots,\mathbf{a}_{m+1}\right\}$也是由$\left\{\mathbf{b}_1,\cdots\mathbf{b}_{m+1}\right\}$线性生成的子空间的正交基。
\end{proof}

特别地,对$n=4$,
\begin{align*}
    \mathbf{a}_1&=\mathbf{b}_1\\
    \mathbf{a}_2&=\mathbf{b}_2-\frac{\left(\mathbf{b}_2|\mathbf{a}_1\right)}{\left(\mathbf{a}_1|\mathbf{a}_1\right)}\mathbf{a}_1\\
    \mathbf{a}_3&=\mathbf{b}_3-\frac{\left(\mathbf{b}_3|\mathbf{a}_1\right)}{\left(\mathbf{a}_1|\mathbf{a}_1\right)}\mathbf{a}_1-\frac{\left(\mathbf{b}_3|\mathbf{a}_2\right)}{\left(\mathbf{a}_2|\mathbf{a}_2\right)}\mathbf{a}_2\\
    \mathbf{a}_4&=\mathbf{b}_4-\frac{\left(\mathbf{b}_4|\mathbf{a}_1\right)}{\left(\mathbf{a}_1|\mathbf{a}_1\right)}\mathbf{a}_1-\frac{\left(\mathbf{b}_4|\mathbf{a}_2\right)}{\left(\mathbf{a}_2|\mathbf{a}_2\right)}\mathbf{a}_2-\frac{\left(\mathbf{b}_4|\mathbf{a}_3\right)}{\left(\mathbf{a}_3|\mathbf{a}_3\right)}\mathbf{a}_3
\end{align*}

\begin{corollary}
    每个有限维赋范内积空间都有一组规范正交基。
\end{corollary}
\begin{proof}
只需要对采用格拉姆--施密特正交化过程得到的正交基,再对其基向量归一化即可。
\end{proof}

\begin{example}[$\mathbb{R}^n$内积空间]
$n$元有序实数对的集合$\mathbb{R}^n$中可定义如下加法运算:
\begin{align*}
\left(x_1,\cdots,x_n\right)^\intercal+\left(y_1,\cdots,y_n\right)^\intercal=\left(x_1+y_1,\cdots,x_n+y_n\right)^\intercal,\\
\alpha\left(x_1,\cdots,x_n\right)^\intercal=\left(\alpha x_1,\cdots,\alpha x_n\right)^\intercal,\forall x_i,y_i,\alpha\in\mathbb{R},i=1,\cdots,n
\end{align*}
可验证,带有上述运算规则的$\mathbb{R}^n$是一个向量空间。再引入如下内积运算:
\[
\left(x_1,\cdots,x_n\right)^\intercal\cdot\left(y_1,\cdots,y_n\right)^\intercal=\sum_{j=1}^nx_jy_j\]
可验证,带有上述运算规则满足内积定义,$\mathbb{R}^n$成变一个内积空间。
\end{example}

\begin{example}
考虑$\mathbb{R}^3$中的三个向量$\mathbf{b}_1=\left(3,0,4\right)^\intercal,\mathbf{b}_2=\left(-1,0,7\right)^\intercal,\mathbf{b}_3=\left(2,9,11\right)^\intercal$。首先可以验证它们线性无关,即
\[
\left\{\begin{array}{rl}
3x_1-x_2+2x_3&=0\\ 
9x_3&=0\\ 
4x_1+7x_2+11x_3&=0
\end{array}\right.
\]
只有唯一解$x_1=x_2=x_3=0$。

通过格拉姆--斯密特正交化过程可由$\mathbf{b}_1,\mathbf{b}_2,\mathbf{b}_3$得到$\mathbf{R}^3$的一组正交基:
\begin{align*}
    \mathbf{a}_1&=\left(3,0,4\right)^\intercal\\
    \mathbf{a}_2&=\left(1,0,7\right)^\intercal-\frac{\left(-1,0,7\right)^\intercal\cdot\left(3,0,4\right)^\intercal}{\left(3,0,4\right)^\intercal\cdot\left(3,0,4\right)^\intercal}\left(3,0,4\right)^\intercal\\
    &=\left(-1,0,7\right)^\intercal-\left(3,0,4\right)^\intercal\\
    &=\left(-4,0,3\right)^\intercal\\
    \mathbf{a}_3&=\left(2,9,11\right)^\intercal-\frac{\left(2,9,11\right)^\intercal\cdot\left(3,0,4\right)^\intercal}{\left(3,0,4\right)^\intercal\cdot\left(3,0,4\right)^\intercal}\left(3,0,4\right)^\intercal-\frac{\left(2,9,11\right)^\intercal\cdot\left(-4,0,5\right)^\intercal}{\left(-4,0,3\right)^\intercal\cdot\left(-4,0,3\right)^\intercal}\left(-4,0,3\right)^\intercal\\
    &=\left(2,9,11\right)^\intercal-2\left(3,0,4\right)^\intercal-\left(-4,0,3\right)^\intercal\\
    &=\left(0,9,0\right)^\intercal
\end{align*}
可见$\mathbf{a}_1,\mathbf{a}_2,\mathbf{a}_3$均为非零向量,故它们是$\mathbb{R}^3$的一组正交基。归一化后得到$\mathbf{\hat{a}}_1=\frac{1}{5}\mathbf{a}_1,\mathbf{\hat{a}}_2=\frac{1}{5}\mathbf{a}_2,\mathbf{\hat{a}}_3=\left(0,1,0\right)^\intercal$是一组规范正交基。任一向量$\mathbf{x}=\left(x_1,x_2,x_3\right)^\intercal\in\mathbb{R}^3$在基$\left\{\mathbf{a}_1,\mathbf{a}_2,\mathbf{a}_3\right\}$下的坐标为:
\[\left(x_1,x_2,x_3\right)^\intercal=\frac{3x_1+4x_3}{25}\mathbf{a}_1+\frac{-4x_1+3x_3}{25}\mathbf{a}_2+\frac{x_2}{9}\mathbf{a}_3\]

由格拉姆--斯密特正交化过程可知,一般地,对于规范正交基$\left\{\mathbf{a}_i\right\}$有
\[\mathbf{a}_i\cdot\mathbf{a}_j=\delta_{ij}\]
其中
\[\delta_{ij}=\left\{\begin{array}{ll}
1,&i=j\\
0,&i\neq j
\end{array}\right.
\]
叫克劳内克符号(Kronecker symbol)。

在选取什么基之下,向量$\mathbf{x}=\left(x_1,\cdots,x_n\right)$在该基下的坐标就恰好是$x_1,\cdots,x_n$呢?这样的基$\left\{\mathbf{\hat{e}}_i\right\}$叫标准基,其中
\[\mathbf{\hat{e}}_i=\left(e_{i,1},\cdots,e_{i,n}\right)^\intercal,e_{i,j}=\delta_{ij}\]
标准基是一个规范正交基。

在选定任一规范正交基$\left\{\hat{\mathbf{a}}\right\}$下,两向量$\mathbf{x}$和$\mathbf{y}\in\mathbb{R}^3$的标准内积计算,按定义有$\mathbf{x}\cdot\mathbf{y}=\sum_ix_iy_i$。另一方面,表达成给定基下的坐标后,则有
\begin{align*}
\mathbf{x}&=\sum_i\mu_i\mathbf{a}_i,\mathbf{y}=\sum_i\nu_i\mathbf{a}_i\\
\mathbf{x}\cdot\mathbf{y}&=\sum_i\mu_i\mathbf{a}_i\cdot\left(\sum_j\nu_j\mathbf{a}_j\right)\\
&=\sum_i\sum_j\left(\mu_i\mathbf{a}_i\right)\cdot\left(\nu_j\mathbf{a}_j\right)\\
&=\sum_i\sum_j\mu_i\nu_j\left(\mathbf{a}_i\cdot\mathbf{a}_j\right)\\
&=\sum_i\sum_j\mu_i\nu_j\delta_{ij}\\
&=\sum_i\mu_i\nu_i
\end{align*}
\end{example}
\end{document}