\documentclass[main.tex]{subfiles}
\begin{document}
本讲义是华南理工大学材料科学与工程学院硕士研究生选修课《聚合物流变学基础》的辅助资料。该课共32学时,面向本科是化学类专业背景的学生。本课拟重点讲述以下内容:

\begin{itemize}
    \item 流变学的连续介质力学基础(包括必要的数学基础)
    \item 线性粘弹性本构方程
    \item 非线性粘弹性本构方程
    \item 流变测量学
\end{itemize}

连续介质力学是流变学的基础。而必要的线性代数和向量函数微积分又是连续介质力学 的数学基础。流变学专注于非线性粘弹性本构关系的研究,是连续介质力学的一个小分支。然而,由于流变学跟高分子科学和工业联系十分紧密,因此高分子学科的学生往往有学习流变学的强烈需求。由于历史的缘故,高分子专业往往是开设在一所大学的化学与化学工程、材料科学与工程等学院,学生普遍不具备学习连续介质力学的数学基础。高分子专业背的学生,要成为一名流变学领域的研究者需要依次跨跃数学、力学和流变学三道槛。

本讲义包括了一个比较详细的数学准备,希望讲义的力学涉及到的数学知识都能找到相关定理和证明。如果由于时间和水平不足、暂未纳入讲义的数学准备部分的数学知识,我也至少会给出参考出处和参考证明。

本讲义的数学准备还尝试在不引入“协变和逆变”的概念的条件下保持自洽性和完整性。

事实上,本讲义数学部分是以Hoffman and Kunze的线性代数和William, Crowell and Trotter的向量函数微积分为原本编写的(见\S \ref{sec:I.3}的建议参考书目)。

本讲义的连续介质力学部分,比某些连续介质力学课本花费了稍多一些的篇幅去讲解事件世界和标架等时空概念。这些较为近代的时空概念,物理专业的学生也许在其他基础力学课程中已经建立了,因此在很多连续介质力学课本中,这类知识多是作为回顾、复习而被简述。但是,要正确理解物理客观性和物质客观性——本构关系的重要要求,必须对不依赖观察者的那些抽象概念十分清楚,而这些是化学专业的学生普遍并不具备的。

在准备此门课的过程中我受到了很多无私的帮助。在此我需重点感谢:我的博士生导师童真教授把我引入到了流变学的世界;李俊杰老师和罗锦添博士牺牲大量私人时间为我回答了大量数学的问题。

\begin{flushright}
孙尉翔\\
2020年10月
\end{flushright}
\end{document}

