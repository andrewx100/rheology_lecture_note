\documentclass[main.tex]{subfiles}
\begin{document}
本讲义是华南理工大学材料科学与工程学院硕士研究生选修课《聚合物流变学基础》的辅助资料。该课共32学时,面向本科是化学类专业背景的学生。本课拟重点讲述以下内容:

\begin{itemize}
    \item 流变学的连续介质力学基础(包括必要的数学基础)
    \item 线性粘弹性本构方程
    \item 非线性粘弹性本构方程
    \item 流变测量学
\end{itemize}

连续介质力学是流变学的基础。而必要的线性代数和向量函数微积分又是连续介质力学 的数学基础。流变学专注于非线性粘弹性本构关系的研究,是连续介质力学的一个小分支。然而,由于流变学跟高分子科学和工业联系十分紧密,因此高分子学科的学生往往有学习流变学的强烈需求。由于历史的缘故,高分子专业往往是开设在一所大学的化学与化学工程、材料科学与工程等学院,学生普遍不具备学习连续介质力学的数学基础。高分子专业背景的学生,要成为一名流变学领域的研究者需要依次跨跃数学、力学和流变学三道槛。

在化学和材料类专业本科的一般课程设置中,学生从《工程力学》、《材料力学》、《化工原理》等课程中接触过的思想,其实都属于连续介质力学。

例如,《工程力学》或《材料力学》一开始就声明了学科的基本假设包括:连续性假设、均匀性假设、各向同性假设和小形变假设。其中的连续性假设,就是连续介质力学的一般性假设。均匀性、各向同性假设使我们可以考虑一个不随位置和方向变化的标量值模量。小形变假设使得课程覆盖的问题均假设虎克弹性。但是,这两门课主要关注的对象是刚体或虎克固体的静力学,既物体处于静止且形变恒定状态下的应力和应变与载荷的关系,因此可进一步利用力系简化原则和平衡条件解题,使得这些题目只需运用标量的代数公式或含简单积分的公式就可以解决。因此,《工程力学》和《材料力学》的知识无法解决更一般的形变问题。其实《工程力学》和《材料力学》概念上还包括运动学和动力学内容,研究的是刚体或虎克固体的运动与形变与载荷的关系,机械能转化效率与机械波的传递等问题,常明显地称作《工程动力学》和《材料动力学》。由于这两门课常常直接基于分析力学和连续介质力学的基础讲述,于是少见于化学和材料类专业的课程中。

又例如,《化工原理》中关于传质部分的内容也声明了流体力学的基本假设:连续性假设和不可压缩假设。事实上,在这一课程中,还默认了流体的均匀性假设和各向同性假设(反过来看,《工程力学》和《材料力学》课程中其实允许了可压缩性,因此介绍了柏松比参数)。流体力学也分为静力学和动力学(称为“流体静力学”和“流体动力学”),二者在《化工原理》课程中都有涉及。这门课关心的对象是理想流体与牛顿流体,原则上既考虑低雷诺数的问题(层流假设)也考虑高雷诺数问题(层流假设失效)。在介绍流体动力学部分时,明确介绍了“质量守恒”和连续性方程,并以直观的方式引入了动量守恒因素,最终推出纳维--斯托克斯方程(Navier--Stokes equation)。这些都是连续介质力学针对牛顿流体的应用。然而,由于《化工原理》的研究目标仅为解决传质传热问题,加上数学基础设置上的限制,《化工原理》课程中的应用例题一般仅限于层流(线性化的纳维--斯托克斯方程)、一维、定态流问题,甚至可以越过纳维--斯托克斯方程的明显引入,直接采用微积分思想列出微分方程解题;往往只关心压差和总流量的关系问题(例如Hagen--Poisseuille方程),避免细化到流速分布场等涉及场函数微积分定理的数学知识的题目,以保持全书篇幅在合理的范围内。

上述两门课中的共同假设就是连续性假设,实际上就是连续介质力学的应用。一般地,连续介质力学在牛顿力学定律的基础上仅需增加这一条假设。但是上述这两门课分别只关心虎克固体和牛顿流体。而流变学关心的非牛顿流动和塑性流动,则属于粘弹性流体或粘塑性流体。这些力学或流动行为,高分子专业的同学在《高分子物理》中也接触过一些基本概念和基本现象,但仅限于应力应变关系,书中没有进一步介绍用于预测和解决实际问题的连续介质力学理论框架。同时对于这些特殊流动现象本身,又往往急于直接从分子层面定性解释,却又略去了基本的统计力学推导。实际上,大部分高分子专业本科的《高分子物理》课程只是一种定性程度的科普。

流变学研究的对象往往被称为复杂流体或软物质(包括但不仅限于高分子)。这类体系的统计力学研究属于(软)凝聚态物理的内容。一般而言,凝聚态物理的研究对像是大量遵循量子力学或经典力学的微观粒子形成的离散的整体(因此并不假定连续性)。这些微观粒子除了满足量子力学或经典力学的假设之外,还需要承认统计力学的公理,包括等概率假设、系综平均的假设和各态遍历假设等等,才能预测系统的整体宏观性质。为了解决复杂的多体相互作用问题,不同体系的统计力学模型还常常引入很多重要的近似思想,如等效介质、平均场、重整化群等,大量运用了近代数学和理论物理的思想和手段,事实上难以完全地嵌入到化学与材料学科的本科教学体系中。

近代软凝聚态物理的发展提供了大量复杂流体体系的微观模型,大大丰富了流变学的应用范畴,因此这些与流变学相关的统计力学模型也常被纳入流变学的教材当中。很多流变学教材都包括了特定体系的统计力学模型的介绍,例如聚合物流变学的内容常常包括从链状分子的动力学出发的流变学模型介绍。流变学需要的连续介质力学和数学基础则常常只在开头花一章左右的篇幅进行很简要的介绍,并在后续章节中少用或不用。至于统计力学模型涉及的更广阔的平衡态统计力学基础,在一个流变学教材的有限篇幅限制内就更无法一一陈列了。专门的《连续介质力学》、《平衡态统计力学》课本又都自成体系,相关的内容常常超出软凝聚态物理和流变学一般关注的范畴(例如连续介质力学中针对晶态固体弹性的的对称性问题、平衡态统计力学常常注重量子统计、电磁场响应等)。这些专门的课程本是物理学专业的学生的必修或选修课程,不能指望大部分化学或材料类专业的学生都去学习。因此,今天的高分子研究面临的尴尬是,高分子专业培养的本科乃至研究生的数理基础无法满足高分子学科前沿研究的需要。高分子专业的价值越来越局限于为已有的高分子工业输出职业人才。事实上,今天写在在高分子物理教科书中的统计物理理论成果,当初都是物理学家做出来的;当时的大学还鲜有专门的高分子专业,高分子问题属于物理学的前沿问题。无奈的是当高分子发展到自己成为独立本科专业的今天,反而发现除高分子专业自己培养的人才之外,已少有理论物理背景的人士加入到高分子物理的研究中来了。这也许是恰恰是因为“高分子学科有自己的本科专业”已成为所有人的印象,其他学科的人鲜有“越殂代疱”者。因此,高分子物理前沿研究的人才,今后一段时间仍需靠高分子专业自己的培养方案来解决。面向高分子背景乃至材料学背景的教科书不仅不应该回避必要的数学和物理基础去作简化,而且还要不照搬面向物理学背景的教材惯例,而重新为化类学生专门去撰写。

综合上述考虑,本讲义仅介绍流变学的连续介质力学基础以及面向化类学生的相关的数学基础。为此本讲义包括了一个比较详细的数学准备,希望讲义的力学涉及到的数学知识都能找到相关定理和证明。如果由于笔者时间和水平不足、暂未纳入讲义的数学准备部分的数学知识,也至少给出证明和推导的参考资料,力求做到全课程中的理论推导不发生中断。本讲义的数学准备还尝试在不引入“协变和逆变”的概念的条件下保持自洽性和完整性,因为针对以流体和无定形固体为主的软物质的特点,可以不引入。这也使得无需正式地介绍张量的概念和高阶张量分析知识,使讲义事实上要求的数学概念不超出本科《高等数学》和《线性代数》的范围过多。具体地,本讲义数学部分是以Hoffman and Kunze的线性代数和William, Crowell and Trotter的向量函数微积分为蓝本编写的(见\S \ref{sec:I.3}的建议参考书目)。

本讲义的连续介质力学部分,比某些连续介质力学课本花费了稍多一些的篇幅去讲解事件世界和标架等时空概念。这些较为近代的时空概念,物理专业的学生也许在其他基础力学课程中已经建立了,因此在很多连续介质力学课本中,这类知识多是作为回顾、复习而被简述。但是,要正确理解物理客观性和物质客观性——本构关系的重要要求,必须对不依赖观察者的那些抽象概念十分清楚,而这些是化学专业的学生普遍并不具备的。

在准备此门课的过程中我受到了很多无私的帮助。在此我需重点感谢:我的博士生导师童真教授把我引入到了流变学的世界;李俊杰老师和罗锦添博士牺牲大量私人时间为我回答了大量数学的问题。

\begin{flushright}
孙尉翔\\
2020年10月
\end{flushright}
\end{document}

