\documentclass[main.tex]{subfiles}
% VI.2.2 向量函数可微分的必要条件与充分条件
\begin{document}
\begin{theorem*}[必要条件之存在性]
定理\ref{thm:II.12.3}:设函数$\mathbf{f}:\mathbb{R}^n\supseteq D\rightarrow\mathbb{R}^m$在点$\mathbf{x}_0\in D$处可微分,即存在线性变换$\mathbf{L}\in\mathcal{L}\left(\mathbb{R}^n,\mathbb{R}^m\right)$满足
\[
\lim_{\Delta\mathbf{x}\to\mathbf{0}}\frac{\mathbf{f}\left(\mathbf{x}_0+\Delta \mathbf{x}\right)-\mathbf{f}\left(\mathbf{x}_0\right)-\mathbf{L}\Delta\mathbf{x}}{\left\|\Delta\mathbf{x}\right\|}=\mathbf{0}
\]
则$\mathbf{f}$的每个坐标函数在$\mathbf{x}_0$处的每个偏导数
\[
\left.\frac{f_i\left(\mathbf{x}\right)}{\partial x_j}\right|_{\mathbf{x}=\mathbf{x}_0},i=1,\cdots,m,j=1,\cdots,n
\]
都存在。若$\left\{\mathbf{\hat{e}}_1,\cdots\mathbf{\hat{e}}_n\right\},\left\{\mathbf{\hat{u}}_1,\cdots,\mathbf{\hat{u}}_m\right\}$分别是$\mathbb{R}^n,\mathbb{R}^m$的标准基,则
\[
\mathbf{L\hat{e}}_j=\sum_{i=1}^m\left(\left.\frac{\partial f_i\left(\mathbf{x}\right)}{\partial x_j}\right|_{\mathbf{x}=\mathbf{x}_0}\right)\mathbf{\hat{u}}_i,j=1,\cdots,n
\]
\end{theorem*}
\begin{proof}
由于函数$\mathbf{f}\left(\mathbf{x}\right)$在$\mathbf{x_0}$处可微分,对任一$j\in\left\{1,\cdots,n\right\}$,令$\Delta\mathbf{x}=t\mathbf{\hat{e}}_j,t\neq0,t\in\mathbb{R}$,则有
\[
\mathbf{f}\left(\mathbf{x}_0+t\mathbf{\hat{e}}_j\right)-\mathbf{f}\left(\mathbf{x}_0\right)=\mathbf{L}\left(t\mathbf{\hat{e}}_j\right)+\left\|t\mathbf{\hat{e}}_j\right\|\mathbf{z}\left(t\mathbf{\hat{e}}_j\right)
\]
其中$\mathbf{z}:\mathbb{R}^n\rightarrow\mathbb{R}^m$满足$\lim_{\mathbf{x}\to\mathbf{0}}\mathbf{z}\left(\mathbf{x}\right)=\mathbf{0}$。上式$\Leftrightarrow$
\begin{align*}
&\lim_{t\to0}\frac{\mathbf{f}\left(\mathbf{x}_0+t\mathbf{\hat{e}}_j\right)-\mathbf{f}\left(\mathbf{x}_0\right)-t\mathbf{L\hat{e}}_j}{t}=\lim_{t\to 0}t\mathbf{z}\left(t\mathbf{\hat{e}}_j\right)=\mathbf{0}\\
\Leftrightarrow&\lim_{t\to0}\frac{\mathbf{f}\left(\mathbf{x}_0+t\mathbf{\hat{e}}_j\right)-f_i\left(\mathbf{x}_0\right)}{t}=\mathbf{L\hat{e}}_j\\
\Leftrightarrow&\lim_{t\to0}\sum_{i=1}^n\frac{f_i\left(\mathbf{x}_0+t\mathbf{\hat{e}}_j\right)-f_i\left(\mathbf{x}_0\right)}{t}\mathbf{\hat{u}}_j=\mathbf{L\hat{e}}_j\\
\Leftrightarrow&\lim_{t\to0}\sum_{i=1}^n\left.\frac{\partial f_i\left(\mathbf{x}\right)}{\partial x_j}\right|_{\mathbf{x}=\mathbf{x}_0}\mathbf{\hat{u}}_j=\mathbf{L\hat{e}}_j\quad\text{(偏导数的定义\ref{def:II.12.4}。)}
\end{align*}
上式对每一$j\in\left\{1,\cdots,n\right\}$都成立。
\end{proof}

\begin{theorem*}[必要条件之唯一性]
定理\ref{thm:II.12.4}:设函数$\mathbf{f}:\mathbb{R}^n\supseteq D\rightarrow\mathbb{R}^m$在点$\mathbf{x}_0\in D$处可微分,即存在线性变换$\mathbf{L}\in\mathcal{L}\left(\mathbb{R}^n,\mathbb{R}^m\right)$满足
\[
\lim_{\Delta\mathbf{x}\to\mathbf{0}}\frac{\mathbf{f}\left(\mathbf{x}_0+\Delta \mathbf{x}\right)-\mathbf{f}\left(\mathbf{x}_0\right)-\mathbf{L}\Delta\mathbf{x}}{\left\|\Delta\mathbf{x}\right\|}=\mathbf{0}
\]
则$\mathbf{L}$是唯一的。
\end{theorem*}
\begin{proof}
设线性变换$\mathbf{L}^\prime\in\mathcal{L}\left(\mathbb{R}^n,\mathbb{R}^m\right)$也满足
\[
\lim_{\Delta\mathbf{x}\to\mathbf{0}}\frac{\mathbf{f}\left(\mathbf{x}_0+\Delta \mathbf{x}\right)-\mathbf{f}\left(\mathbf{x}_0\right)-\mathbf{L}^\prime\Delta\mathbf{x}}{\left\|\Delta\mathbf{x}\right\|}=\mathbf{0}
\]
令$\mathbf{A}=\mathbf{L}-\mathbf{L}^\prime$,则
\begin{align*}
   &\mathbf{A}\Delta\mathbf{x}=\mathbf{f}\left(\mathbf{x}+\Delta\mathbf{x}\right)-\mathbf{f}\left(\mathbf{x}\right)-\mathbf{L}\Delta\mathbf{x}-\left[\mathbf{f}\left(\mathbf{x}+\Delta\mathbf{x}\right)-\mathbf{f}\left(\mathbf{x}\right)-\mathbf{L}^\prime\left(\Delta\mathbf{x}\right)\right]\\
   \Rightarrow&\left\|\mathbf{A}\Delta\mathbf{x}\right\|\leq\left\|\mathbf{f}\left(\mathbf{x}+\Delta\mathbf{x}\right)-\mathbf{f}\left(\mathbf{x}\right)-\mathbf{L}\Delta\mathbf{x}\right\|+\left\|\mathbf{f}\left(\mathbf{x}+\Delta\mathbf{x}\right)-\mathbf{f}\left(\mathbf{x}\right)-\mathbf{L}^\prime\Delta\mathbf{x}\right\|\quad\text{(三角不等式。)}\\
   \Rightarrow&\lim_{\Delta\mathbf{x}\to\mathbf{0}}\frac{\left\|\mathbf{A}\Delta\mathbf{x}\right\|}{\left\|\Delta\mathbf{x}\right\|}=0\quad\text{(夹逼定理。)}\\
\end{align*}
由于$\Delta\mathbf{x}\to0$的路径不限,不仿考虑固定$\Delta\mathbf{x}\neq\mathbf{0}$,下式亦成立
\[\lim_{t\to0}\frac{\left\|\mathbf{a}\left(t\Delta\mathbf{x}\right)\right|}{\left|t\Delta\mathbf{x}\right\|}=0\Leftrightarrow\lim_{t\to0}\frac{\left\|\mathbf{A}\Delta\mathbf{x}\right\|}{\left\|\Delta\mathbf{x}\right\|}=0\Rightarrow\mathbf{A}=\mathbf{0}\]
\end{proof}

\begin{theorem*}[充分条件]
定理\ref{thm:II.12.5}:若函数$\mathbf{f}:\mathbb{R}^n\supset D\rightarrow\mathbb{R}^m$的定义域$D$是开集,偏微分$\frac{\partial f_i}{\partial x_j},i=1,\cdots,n,j=1,\cdots,m$在$D$内都连续,则$\mathbf{f}$在$D$内均可微分。
\end{theorem*}
\begin{proof}
设$\mathbf{x}=\left(b_1,\cdots,b_n\right)^\intercal,\mathbf{x}_0=\left(a_1,\cdots,a_n\right)^\intercal\in D$,令$\mathbf{y}_k=\left(b_1,\cdots,b_k,a_{k+1},\cdots,a_n\right),k=0,\cdots,n$,则有$\mathbf{y}_0=\mathbf{x}_0,\mathbf{y}_n=\mathbf{x}$,且$\left\|\mathbf{y}_k-\mathbf{y}_{k-1}\right\|=\left|b_k-a_k\right|,k=1,\cdots,n$。故有$\mathbf{f}\left(\mathbf{x}\right)-\mathbf{f}\left(\mathbf{x}_0\right)=\sum_{k=1}^n\left(\mathbf{f}\left(\mathbf{y}_k\right)-\mathbf{f}\left(\mathbf{y}_{k-1}\right)\right),k=1,\cdots,n$。考察上式等号右边的求和项:
\[\mathbf{f}\left(\mathbf{y}_k\right)-\mathbf{f}\left(\mathbf{y}_{k-1}\right)=\sum_{j=1}^m\left(f_j\left(\mathbf{y}_k\right)-f_j\left(\mathbf{y}_{k-1}\right)\right)\mathbf{\hat{e}}_j
\]
其中$\left\{\mathbf{\hat{e}}_i\right\}$是$\mathbb{R}^n$的标准基。注意到点$\mathbf{y}_k$与$\mathbf{y}_{k-1}$之间的连线是长度为$\left|b_k-a_k\right|$、方向与第$k$个坐标轴$\mathbf{\hat{e}}_k$平行的有向线段。

对上式右边的坐标函数应用微分中值定理。由命题条件,坐标函数的偏导数$\frac{\partial f_i}{\partial x_j}$在$D$内都连续,则坐标函数$f_i$本身在$D$内也连续,在$\mathbf{y}_k$与$\mathbf{y}_{k-1}$连线上必存在一点$c_k$使得
\[\frac{f_i\left(\mathbf{y}_k\right)-f_i\left(\mathbf{y}_{k-1}\right)}{b_k-a_k}=\left.\frac{\partial f_i}{\partial x_k}\right|_{x_k=c_k}\]
代入上一个求和式得:
\begin{align*}
\mathbf{f}\left(\mathbf{y}_k\right)-\mathbf{f}\left(\mathbf{y}_{k-1}\right)&=\sum_{j=1}^m\left.\frac{\partial f_j}{\partial x_k}\right|_{x_k=c_k}\left(b_k-a_k\right)\mathbf{\hat{e}}_j\\
&=\left(b_k-a_k\right)\left.\frac{\partial \mathbf{f}}{\partial x_k}\right|_{x_k=c_k},k=1,\cdots,n
\end{align*}
再代入上一个求和式得:
\[\mathbf{f}\left(\mathbf{x}\right)-\mathbf{f}\left(\mathbf{x}_0\right)=\sum_{k=1}^n\left(b_k-a_k\right)\left.\frac{\partial \mathbf{f}}{\partial x_k}\right|_{x_k=c_k}
\]

令线性变换$\mathbf{L}:\mathbb{R}^n\rightarrow \mathbb{R}^m$满足
\begin{align*}
\left(\mathbf{L}\left(\mathbf{x}-\mathbf{x}_0\right)\right)&=\left(\left(\left.\frac{\partial \mathbf{f}}{\partial x_1}\right|_{x_1=a_1}\right),\cdots,\left(\left.\frac{\partial \mathbf{f}}{\partial x_1}\right|_{x_n=a_n}\right)\right)\left(b_1-a_1,\cdots,b_n-a_n\right)^\intercal\\
&=\sum_{k=1}^n\left(x_k-a_k\right)\left.\frac{\partial \mathbf{f}}{\partial x_k}\right|_{x_k=a_k}
\end{align*}
则有
\begin{align*}
    \left\|\mathbf{f}\left(\mathbf{x}\right)-\mathbf{f}\left(\mathbf{x}_0\right)-\mathbf{L}\left(\mathbf{x}-\mathbf{x}_0\right)\right\|&=\left\|\sum_{k=1}^n\left(\left.\frac{\partial \mathbf{f}}{\partial x_k}\right)|_{x_k=c_k}-\left.\frac{\partial \mathbf{f}}{\partial x_k}\right|_{x_k=a_k}\right)\left(x_k-a_k\right)\right\|\\
    &\leq\sum_{k=1}^n\left\|\left.\frac{\partial \mathbf{f}}{\partial x_k}\right|_{x_k=c_k}-\left.\frac{\partial\mathbf{f}}{\partial x_k}\right|_{x_k=a_k}\right\|\left|x_k-a_k\right|\\
    &\leq\sum_{k=1}^n\left\|\left.\frac{\partial\mathbf{f}}{\partial x_k}\right|_{x_k=c_k}-\left.\frac{\partial\mathbf{f}}{\partial x_k}\right|_{x_k=a_k}\right\|\left\|\mathbf{x}-\mathbf{x}_0\right\|
\end{align*}
由于上述的不等式总成立,则有当$\mathbf{x}\to\mathbf{x}_0$时$c_k\to a_k$。又由命题条件$\frac{\partial f_i}{\partial x_j}$在$D$都连续,即极限
\[
\lim_{x_j\to a_k}\frac{\partial f_j}{\partial x_j}=\left.\frac{\partial f_i}{\partial x_j}\right|_{x_j=a_k},j=1,\cdots,n
\]
都存在,故以下极限等式成立:
\[\lim_{\mathbf{x}\to\mathbf{x}_0}\frac{\mathbf{f}\left(\mathbf{x}\right)-\mathbf{f}\left(\mathbf{x}_0\right)-\mathbf{L}\left(\mathbf{x}-\mathbf{x}_0\right)}{\left\|\mathbf{x}-\mathbf{x}_0\right\|}=\mathbf{0}
\]
由全微分的定义,命题得证,且线性变换$\mathbf{L}$就是函数的全导数。
\end{proof}
\end{document}