\documentclass[main.tex]{subfiles}
\begin{document}
力学定律离不开对物体空间位置及其变化的描述。在经典力学里,我们认为物体是在三维欧几里德空间中运动的。这里的“三维区几里德空间”是需要定义。

作为出发点,我们先引入欧几里德空间的最传统的定义,即“由欧几里德几何公设所描述或建立的空间”,这种空间的几何学是不依赖坐标系就可以被写下(written down)的,称为“综合几何”(synthetic geometry)。相比之下,解析几何(analytic geometry)则总是需要在采用了适当的坐标系之后,才可以被写下。综合几何仅依赖少数的几个“原始概念”(primitive notions)。一般来说,原始概念既包括某些对象(例如“点”、“直线”、“平面”)、又包括它们的关系(例如“相交)。除了原始概念外,综合几何还依赖关于这些原始概念的公设(axioms)。公设是无法被证明、假设总为真的陈述。例如“过两点有且仅有一条直线”。欧几里德的《几何原本》的几何部分(第一至六卷),就是一套综合几何。在欧几里德几何里,还有一套证明性的“工具”,即尺规作图,即仅用理想直尺和理想圆规作出的图形。因为尺规作图是符合欧几里德公设的,所以尺规作图成了欧几里德几何的证明工具。

\footnote{欧几里德的《几何原本》第一至六卷是几何部分。在美国克拉克大学网站上有英文原文:https://mathcs.clarku.edu/~djoyce/java/elements/toc.html}。
\end{document}