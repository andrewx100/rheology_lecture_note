\documentclass[main.tex]{subfiles}
\begin{document}
我们将之前关于线性变换的部分定理再次总结如下:
\begin{theorem}
对于一个线性变换$\mathbf{T}:\mathcal{V}\rightarrow\mathcal{W}$,若$\mathcal{V}$与$\mathcal{W}$的维数相同,则以下等价命题皆成立:
\begin{itemize}
    \item $\mathbf{T}$是同构映射
    \item $\mathbf{T}$可逆
    \item 如果$\left\{\mathbf{e}_i\right\}$是$\mathcal{V}$的基,则$\mathbf{T}\left\{\mathbf{e}_i\right\}$是$\mathcal{W}$的基
    \item 对$\mathcal{W}$的一组基$\left\{\mathbf{e}^\prime_j\right\}$,总能从$\mathcal{V}$中找到相应的一组基$\left\{\mathbf{e}_i\right\}$使得$\mathbf{e}^\prime_j=\mathbf{T}\mathbf{e}_i$
\end{itemize}
\label{theo:linear_operator}
\end{theorem}

现在,我们引入本讲义内张量的定义\footnote{正式的张量分析课本中,张量和张量积有更加一般的定义。本讲义定义的概念、记法和定理,仅适用于本讲义。}。
\begin{definition}
在本讲义范围内,二阶张量(简称张量)就是从$\mathbb{R}$上的$N$维向量空间$\mathcal{V}_N$到其自身的线性变换(即线性算符);一阶张量就是属于$\mathcal{V}_N$的向量;零阶张量是属于$\mathbb{R}$的标量。
\end{definition}

易知,二阶张量具有定理\ref{theo:linear_operator}中的所有性质。

现在,我们引入张量积的定义。
\begin{definition}[两个向量的张量积]
两个向量$\mathbf{v},\mathbf{w}\in\mathcal{V}$的张量积$\mathbf{v}\otimes\mathbf{w}\in\mathcal{L}\left(\mathcal{V}\right)$,是一个线性二阶张量满足:
\[\left(\mathbf{v}\otimes\mathbf{w}\right)\mathbf{u}=\mathbf{v}\left(\mathbf{w}\cdot\mathbf{u}\right),\forall\mathbf{u}\in\mathcal{v}\]
\end{definition}

易验证张量积满足线性变换的代数规则。

并非每个张量都能表示成两个向量的向量积。我们把$\mathcal{V}$的两个向量的向量积的集合记为$\mathcal{V}\otimes\mathcal{V}$,由张量积的定义易知,$\mathcal{V}\otimes\mathcal{V}$是向量空间。因此$\mathcal{V}\otimes\mathcal{V}$是$\mathcal{L}\left(\mathcal{V}\right)$的子空间。

\begin{theorem}
给定向量空间$\mathcal{V}$的一组基$\left\{\mathbf{e}_i\right\}\in\mathcal{V}$,其基向量的所有张量积$\left\{\mathbf{e}_i\otimes\mathbf{e_j}\right\}$组成$\mathcal{V}\otimes\mathcal{V}$的一组基。
\end{theorem}

\begin{theorem}
$N$维向量的张量积组成的空间是$\mathcal{V}_N\otimes\mathcal{V}_N$是$N\times N$维向量空间。
\end{theorem}

又因为$\mathcal{L}\left(\mathcal{V}_N\right)$在给定任一基下与$N$维方阵$\mathcal{M}^{N\times N}$空间是同构的。因此为$\mathcal{L}\left(\mathcal{V}_N\right)$是$N\times N$维向量空间。此外易证(故未作为定理列出),若向量空间的维数与它的一个子空间的维数相同,则这个向量空间就是这个子空间的任一组基的线性生成空间。因此$\left\{\mathbf{e}_i\otimes\mathbf{e}_j\right\}$也是$\mathcal{L}\left(\mathcal{V}\right)$的一组基。

我们可以按照向量在给定基下的坐标的表示法去表示两个向量的张量积或一个张量。

\begin{example}
设向量$\mathbf{v},\mathbf{w}\in\mathcal{V}_N$且$\left\{\mathbf{e}_i\right\}$是$\mathcal{V}_N$的一组基,张量$\mathbf{A}\in\mathcal{L}\left(\mathcal{V}_N\right)$,则有:
\[\mathbf{v}=\sum_{i=1}^Nv_i\mathbf{e}_i,\mathbf{w}=\sum_{i=1}^Nw_i\mathbf{e}_i\]
\begin{align*}
\mathbf{v}\otimes\mathbf{w}&=\left(\sum_{i=1}^Nv_i\mathbf{e}_i\right)\otimes\left(\sum_{j=1}^Nw_j\mathbf{e}_j\right)\\&=\sum_{i=1}^N\sum_{j=1}^Nv_iw_j\mathbf{e}_i\otimes\mathbf{e}_j\\
\mathbf{A}&=\sum_{i=1}^N\sum_{j=1}^NA_{ji}\mathbf{e}_j\otimes\mathbf{e}_i
\end{align*}

若$\left(\mathbf{A}\right)=\left(\alpha_{ji}\right)$为张量$\mathbf{A}$在基$\left\{\mathbf{e}_i\right\}$下的表示矩阵,且$\mathbf{A}=\sum_{i=1}^N\sum_{j=1}^NA_{ji}\mathbf{e}_j\otimes\mathbf{e}_i$,则当且仅当$\left\{\mathbf{e}_i\right\}$为规范正交基时有$A_{ji}=\alpha_{ji}$。
\end{example}

\begin{definition}
两个张量的复合积就是它们作为两个线性变换的复合变换。
\end{definition}

\begin{definition}[张量的转置]
$\mathbf{A}$的转置$\mathbf{A}^\intercal$是$\mathbf{A}$在选定某一基下的矩阵的转置所对应的张量\footnote{这与一般的线性变换的转置定义并不相容。一般地,两个向量空间的线性变换的转置是这两个向量空间的对偶空间之间的线性变换,属于与原线性变换不同的空间。但此处定义的张量的转置与原张量属于同一个空间。这一矛盾是由本讲义直接把张量定义为线性算子造成的。为了维持“张量是线性变换”这种定义,本讲义正文中不提线性变换的转置,以回避在转置定义上的矛盾。}。
\end{definition}

\begin{definition}[两个张量的标量积]
两个张量$\mathbf{A},\mathbf{B}\in\mathcal{L}\left(\mathcal{V}\right)$的内积或标量积,记为$\mathbf{A}:\mathbf{B}$,是一个村量。这一运算是一个由$\mathcal{L}\left(\mathcal{V}\right)$到$\mathbb{R}$的映射,定义为:$\mathbf{A}:\mathbf{B}=\mathrm{tr}\left(\mathbf{A}^\intercal\mathbf{B}\right)$。
\end{definition}

\begin{definition}[张量的逆]
张量的可逆性与逆的定义,与线性变换的相同。
\end{definition}

\begin{definition}[对称张量与斜称张量]
如果张量$\mathbf{A}=\mathbf{A}^\intercal$,则称$\mathbf{A}$是对称张量。如果$\mathbf{A}^\intercal=-\mathbf{A}$,则称$\mathbf{A}$是斜称张量。
\end{definition}

\begin{theorem}
任何一个张量$\mathbf{A}$均可以写成一个对称张量$\mathbf{A}^\mathrm{s}$与一个斜称张量$\mathbf{A}^\mathrm{ss}$之和:
\[\mathbf{A}=\mathbf{A}^\mathrm{s}+\mathbf{A}^\mathrm{ss}\]
其中
\[\begin{split}\mathbf{A}^\mathrm{s}&=\frac{1}{2}\left(\mathbf{A}+\mathbf{A}^\intercal\right)\\\mathbf{A}^\mathrm{ss}&=\frac{1}{2}\left(\mathbf{A}-\mathbf{A}^\intercal\right)\end{split}\]
\end{theorem}
\end{document}