\documentclass[main.tex]{subfiles}
\begin{document}
\begin{definition}[代数]
若$\mathcal{A}$是$\mathbb{F}$上的向量空间,$\mathcal{A}$含有如下定义的运算
\[\mu:\mathcal{V}\times\mathcal{V}\rightarrow\mathcal{V},\mu\left(\mathbf{a},\mathbf{b}\right)=\mathbf{a}\times\mathbf{b}\]
满足
\[\mathbf{a}\times\left(\beta\mathbf{b}+\gamma\mathbf{c}\right)=\beta\left(\mathbf{a}\times\mathbf{b}\right)+\gamma\left(\mathbf{a}\times\mathbf{c}\right),\forall\mathbf{a},\mathbf{b},\mathbf{c}\in\mathcal{A},\beta,\gamma\in\mathbb{F}\]
则称$\mathcal{A}$是数域$\mathbb{F}$上的一个代数。

如果$\mathcal{A}$满足$\mathbf{a}\times\left(\mathbf{b}\times\mathbf{c}\right)=\left(\mathbf{a}\times\mathbf{b}\right)\times\mathbf{c}$则称$\mathcal{A}$满足结合率。

如果$\mathcal{A}$满足$\mathbf{a}\times\mathbf{b}=\mathbf{b}\times\mathbf{a}$则称$\mathcal{A}$满足交换率。

如果$\mathcal{A}$含有元素$\bm{1}$满足$\bm{1}\times\mathbf{a}=\mathbf{a}\times\bm{1}=\mathbf{a}$,则称$\mathcal{A}$具有单位元。

如果$\mathcal{A}$具有单位元,$\mathbf{b}\times\mathbf{a}=\bm{1}$,则称$\mathbf{b}$是$\mathbf{a}$的左逆,$\mathbf{a}$是$\mathbf{b}$的右逆。

给定$\mathcal{V}_N$的一组基$\mathcal{B}=\{\hat{\mathbf{e}}_i\}_{i=1}^N$和一个代数$\mathcal{A}$,则
\[\hat{\mathbf{e}}_i\times\hat{\mathbf{e}}_j=\sum_{i=1}^Nc_{ij}^k\hat{\mathbf{e}}_k,c_{ij}^k\in\mathbb{F}\]
其中$c_{ij}^k$称为代数$\mathcal{A}$的结构常数。
\end{definition}

简而言之,对集合的两元素进行代数运算的结果仍属于原集合。例:
\begin{itemize}
    \item $\mathbb{F}$上的向量空间都是$\mathbb{F}$上的代数,零向量$\bm{0}$是单位元。
    \item 复数域$\mathbb{C}$是实数域$\mathbb{R}$上的一个代数。
\end{itemize}

\begin{definition}[向量的叉乘]
如果$\mathbb{R}$上的代数$\mathcal{A}$在其规范正交基$\{\hat{\mathbf{e}}_i\}_{i=1}^N$下的结构常数为列维--奇维塔符号
\[\varepsilon_{i_1\cdots i_N}=\prod_{p>q}\frac{i_p-i_q}{\left|i_p-i_q\right|}\]
则代数$\mathcal{A}$的向量乘法称为叉乘。
\end{definition}
特别地,当$N=3$时,
\[\varepsilon_{ij}^k=
\begin{cases}
0,&i=j\text{或}j=k\text{或}i=k\\
1,&\left(i,j,k\right)\in\{\left(1,2,3\right),\left(2,3,1\right),\left(3,1,2\right)\}\\
-1,&\left(i,j,k\right)\in\{\left(1,3,2\right),\left(3,2,1\right),\left(2,1,3\right)\}\end{cases}\]
由$\varepsilon_{ij}^k=-\varepsilon_{ji}^k$等性质可验证,$\forall\mathbf{a},\mathbf{b},\mathbf{c}\in\mathcal{A}_3,\lambda\in\mathbb{F}$,\begin{itemize}
    \item $\mathbf{a}\times\mathbf{b}=-\mathbf{b}\times\mathbf{a}$
    \item $\mathbf{a}\times\mathbf{a}=\bm{0}$
    \item $\left(\lambda\mathbf{a}\right)\times\mathbf{b}=\mathbf{a}\times\left(\lambda\mathbf{b}\right)=\lambda\left(\mathbf{a}\times\mathbf{b}\right)$
    \item $\left(\mathbf{a}+\mathbf{b}\right)\times\mathbf{c}=\mathbf{a}\times\mathbf{c}+\mathbf{b}\times\mathbf{c}$
    
\end{itemize}
\end{document}